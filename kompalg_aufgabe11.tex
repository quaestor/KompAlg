\section*{Aufgabe 11 - Douglas Hofstadters MIU-System}
\addcontentsline{toc}{subsection}{Aufgabe 11 - Douglas Hofstadters MIU-System}
\begin{itemize}
  \item $MIUI$ lässt sich erzeugen:
  \[ MI \vdash_{2} MII \vdash_{2} MIIII \vdash_{2} MIIIIIIII \vdash_{3} \]
  \[ MIUIIII \vdash_{1} MIUIIIIU \vdash_{3} MIUIUU \vdash_{4} MIUI \]
  
  $MU$ lässt sich nicht erzeugen, da man $3\cdot a\ (a \in \mathds{N})$ '$I$'
  benötigt, um $U$ ohne verbleibende $I$ zu erzeugen, was nicht möglich ist, da
  man nur $2^{n}\ (n \in \mathds{N})$ '$I$' generieren kann.

  \item Beginn mit {\textit Axiom} $MI$
    \begin{itemize}
      \item Durch Regel $2$ erzeugt man $2^{n}$ '$I$'. $2^{n}$ ist nicht durch
      $3$ teilbar. Durch Regel $3$ erhält man $2^{n} - 3a$ '$I$' (ebenfalls
      nicht durch $3$ teilbar).
      \item Wird nach Regel $3$ wieder Regel $2$ angewendet, so enthält das
      Wort $2(2^{n} - 3a)$ '$I$'. Da der geklammerte Term nicht durch $3$
      teilbar ist, kann auch das Doppelte davon nicht durch $3$ teilbar sein.
      \item Es gibt keine Regel, durch die ein weiteres '$M$' erzeugt oder das
      vorhandene entfernt wird.
    \end{itemize}
    Daraus folgt die Behauptung.

  \item Gegeben sei oBdA das Wort $Mx$ mit $x=I^{i+3u}$, wobei $i$ die Anzahl
  der '$I$' und $u$ die Anzahl der 'U' ist. Dabei sei $3\not|\ i$, da sich $3$
  '$I$' zu '$U$' zusammenfassen lassen (Regel $3$) und somit zum Beweis ein
  Wort ohne $U$ ausreicht.
  \[ k := i + 3u \]
  \begin{itemize}
    \item Aus Regel $2$ folgt $\hat x \vdash \hat x^{2}$
    \item Aus Regeln $1, 3, 4$ folgt $\hat xIII \vdash \hat x$
    \item \textsc{Achtung:} \textbf{eventuell noch fehlerhaft!}
    \item Da $k \equiv 2^{n} (mod 3) \Rightarrow 2^{n} - 3a 
                = k\quad a \in \mathds{N}^{+}$
    \item[$\Rightarrow$] $MI^{2^{n}} \vdash^{*} MI^{k}$
  \end{itemize}

  \item Automat:

    \begin{tikzpicture}[shorten >=1pt,node distance=2cm,on grid,auto]
      \node[state,initial]    (S)                {$S$};
      \node[state]            (Z1) [right=of S]  {$z_{1}$};
      \node[state,accepting]  (a2) [right=of Z1] {$z_{2}$};
      \node[state,accepting]  (Z3) [right=of a2] {$z_{3}$};
      \path[->] (S)   edge              node {$M$} (Z1)
                (Z1)  edge [loop above] node {$U$} ()
                      edge              node {$I$} (a2)
                (a2)  edge [loop above] node {$U$} ()
                      edge              node {$I$} (Z3)
                (Z3)  edge [loop above] node {$U$} ()
                      edge [bend left]  node {$I$} (Z1);
    \end{tikzpicture}

    Transfer-Matrix:
    \[ M = \begin{bmatrix} 1 & 1 & 0 \\ 0 & 1 & 1 \\ 1 & 0 & 1 \end{bmatrix} \]
    Ausdruck:
    \[ M(U+IU^{*}IU^{*}I)^{*}(IU^{*}IU^{*}+IU^{*}) \]

  \item Die Startwerte bestimmt man aus dem regulären Audruck. Das Wort hat
    mindestens Länge $2$:
    \[ \Rightarrow \mu_{0} = \mu_{1} = 0 \]
    Mit Länge $2$ ist nur $MI$ möglich:
    \[ \Rightarrow \mu_{2} = 1 \]
    Charakteristisches Polynom von $M$:
    \[ \chi_{M}(\lambda) = det(M - \lambda \cdot \mathds{1})
       = \begin{vmatrix} 1-\lambda & 1 & 0 \\ 
                         0 & 1-\lambda & 1 \\ 
                         1 & 0 & 1-\lambda \end{vmatrix} = \]
    \[ = (1 - \lambda)^3 + 1 = 1 - 3\lambda + 3\lambda^2 - \lambda^3 + 1 = \]
    \[ = -\lambda^{3} + 3\lambda^{2} - 3\lambda + 2 \]

    \[ \lambda^{3} = 3\lambda^{2} - 3\lambda + 2 \Leftrightarrow \lambda^{n}
       = 3\lambda^{n-1} - 3\lambda^{n-2} + 2\lambda^{n-3} \]
    \[ \Rightarrow \mu_{n} = 3\mu_{n-1} - 3\mu_{n-2} + 2\mu_{n-3}
       \quad (n \geq 3) \]

  \item Eigenwerte:
    \[ -\lambda^{3} + 3\lambda^{2} - 3\lambda + 2 = 0\]
    \[ (\lambda - 2) (\lambda^{2} - \lambda + 1) 
       = (\lambda - 2) (\lambda - \alpha) (\lambda - \hat \alpha) = 0\]
    \[ \Rightarrow \lambda = 2, 
       \alpha = \frac{1+i\sqrt{3}}{2}, 
       \hat \alpha = \frac{1-i\sqrt{3}}{2} \]
    $\alpha$ und $\hat \alpha$ sind sechste Einheitswurzeln $\Rightarrow
    |\alpha| = |\hat \alpha| = 1$

  \item Bestimmung der Koeffizienten über \textsc{Vandermonde}-Matrix:
    \[ \begin{pmatrix} a&b&c \end{pmatrix} 
       \begin{pmatrix} 1 & \lambda & \lambda^{2} \\ 
                       1 & \alpha  & \alpha^{2} \\
                       1 & \hat \alpha & \hat \alpha^{2} \end{pmatrix} =
       \begin{pmatrix} 0 & 0 & 1 \end{pmatrix} \]
    ergibt das Gleichungssystem
    \begin{eqnarray*}
    a + b + c &=& 0 \\
    2a + \alpha b + \hat \alpha c &=& 0 \\
    4a + \alpha^{2} b + \hat \alpha^{2} c &=& 1
    \end{eqnarray*}
    mit den Lösungen (unter Zuhilfenahme der Tatsache dass $\hat \alpha = (1 -
    \alpha)$ sowie $\alpha^{2} = \alpha - 1$ und $\hat \alpha^{2} = \alpha$)
    \[ a = \frac{1}{3}, \quad
       b = -\frac{1}{3}\hat \alpha, \quad
       c = -\frac{1}{3}\alpha \]

  \item Da $\alpha$ und $\hat \alpha$ den Absolutbetrag $1$ haben wirkt sich
    $b\alpha^{n} + c\hat \alpha^{n}$ nicht auf das asymptotische Verhalten aus.
    \[ \Rightarrow \mu_{n} \sim a\cdot 2^{n} = \frac{2^{n}}{3} \]

  \item (Optionale Aufgabe) -- später...
\end{itemize}
