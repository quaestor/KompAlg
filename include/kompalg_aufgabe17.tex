\section*{Aufgabe 17 - inhomogene Rekursion}
\addcontentsline{toc}{subsection}{Aufgabe 17 - inhomogene Rekursion}
\[ x_{n} = 4x_{n-1} - 4x_{n-2} + f_{n} \quad (n\geq 2) \]
\begin{paragraph}{1)}
  Lösungen der homogenen Rekursion
  \[ x_{n} = 4x_{n-1} - 4x_{n-2} \]
  \[ \chi(z) = z^{2} - 4z + 4 = (z - 2)^{2} \]
  also der doppelte Eigenwert
  \[ \lambda = 2 \]
  und damit
  \[ x_{n} = \alpha_{1}2^{n} + \alpha_{2}n2^{n} \]

  \[ \textnormal{für } \alpha_{2} = 0 \Rightarrow x_{n} \sim \alpha_{1}2^{n} \]
  \[ \textnormal{für } \alpha_{2} \not= 0 \Rightarrow x_{n} \sim \alpha_{2}n2^{n} \]
  \[ \textnormal{für } x_{0} = x_{1} = 0 \Rightarrow x_{n} = 0 \]
\end{paragraph}
\begin{paragraph}{2)}
  spezielle Lösungen der inhomogenen Rekursion
  \begin{enumerate}[(a)]
    \item $f = (1, 1, \dots) = (1^{n})_{n\geq 0}$
      \[ x_{n} = 4x_{n-1} - 4x_{n-2} + 1^{n} \]
      \[ x_{n+1} = 4x_{n} - 4x_{n-1} + 1^{n+1} \]

      \[ x_{n} - x_{n+1} = 4x_{n-1} - 4x_{n-2} - 4x_{n} + 4x_{n-1} \]
      \[ -x_{n+1} = -5x_{n} + 8x_{n-1} - 4x_{n-2} \]
      \[ x_{n} = 5x_{n-1} - 8x_{n-2} + 4x_{n-3} \]

      \[ \chi(z) = z^{3} - 5z^{2} + 8z -  4 = (z - 1) (z - 2)^{2} \]
      \[ \lambda_{1} = 1, \lambda_{2} = 2 \textnormal{ (doppelter EW)} \]

      \[ x_{n} = \alpha_{1}1^{n} + \alpha_{2}2^{n} + \alpha_{3}n2^{n} \]
      Durch einsetzen in die ursprüngliche Rekursion erhält man $\alpha_{1} =
      1$ und damit
      \[ x_{n} = 1 + \alpha 2^{n} + \beta n2^{n} \]
    \item $f = (1, 2, 4, \dots) = (2^{n})_{n\geq 0}$
      \[ x_{n} = 4x_{n-1} - 4x_{n-2} + 2^{n} \]
      \[ x_{n+1} = 4x_{n} - 4x_{n-1} + 2^{n+1} \]

      \[ 2x_{n} - x_{n+1} = 8x_{n-1} - 8x_{n-2} - 4x_{n} + 4x_{n-1} \]
      \[ -x_{n+1} = -6x_{n} + 12x_{n-1} - 8x_{n-2} \]
      \[ x_{n} = 6x_{n-1} - 12x_{n-2} + 8x_{n-3} \]

      \[ \chi(z) = z^{3} - 6z^{2} + 12z - 8 = (z - 2)^{3} \]
      \[ \lambda = 2, \textnormal{ (dreifacher EW)} \]

      \[ x_{n} = \alpha_{1}2^{n} + \alpha_{2}n2^{n} + \alpha_{3}n^{2}2^{n} \]
      Durch einsetzen in die ursprüngliche Rekursion erhält man wiederum
      $\alpha_{3} = \frac{1}{2}$ und damit
      \[ x_{n} = \alpha_1 2^{n} + \alpha_2 n2^{n} + n^{2}2^{n-1}\]
    \item $f = (1, 3, 9, \dots) = (3^{n})_{n\geq 0}$
      \[ x_{n} = 4x_{n-1} - 4x_{n-2} + 3^{n} \]
      \[ x_{n+1} = 4x_{n} - 4x_{n-1} + 3^{n+1} \]

      \[ 3x_{n} - x_{n+1} = 12x_{n-1} - 12x_{n-2} - 4x_{n} + 4x_{n-1} \]
      \[ -x_{n+1} = -7x_{n} + 16x_{n-1} - 12x_{n-2} \]
      \[ x_{n} = 7x_{n-1} - 16x_{n-2} + 12x_{n-3} \]

      \[ \chi(z) = z^{3} - 7z^{2} + 16z - 12 = (z - 3) (z - 2)^{2} \]
      \[ \lambda_{1} = 3, \lambda_{2} = 2, \textnormal{ (doppelter EW)} \]

      \[ x_{n} = \alpha_{1}3^{n} + \alpha_{2}2^{n} + \alpha_{3}n2^{n} \]
      Erneut wird durch Einsetzen in die ursprüngliche Rekursionsgleichung der
      Parameter $\alpha_{1} = 9$ eliminiert und es ergibt sich
      \[ x_{n} = \alpha 2^{n} + \beta n2^{n} + 3^{n+2} \]
  \end{enumerate}
  \begin{subparagraph}{Asymptotisches Verhalten}
    \begin{enumerate}[(a)]
      \item \[ x_{n} = 1 + \alpha 2^{n} + \beta n2^{n} \]
        \begin{equation*}
          \begin{split}
           \beta \not= 0 &\quad x_{n} \sim \beta n2^{n} \\
           \beta = 0     &\quad x_{n} \sim \alpha 2^{n}
          \end{split}
        \end{equation*}
      \item \[ x_{n} = \alpha2^{n} + \beta n2^{n} + n^{2}2^{n-1}\]
        \[ x_{n} \sim n^{2}2^{n-1} \]
      \item \[ x_{n} = \alpha 2^{n} + \beta n2^{n} + 3^{n+2} \]
        \[ x_{n} \sim 3^{n+2} \]
    \end{enumerate}
  \end{subparagraph}
  \begin{subparagraph}{Alternative Herangehensweise}
	Alternativ lässt sich diese Aufgabe auch durch Überlegungen aus dem
	Skript lösen:
	\[ x_{n} = 4x_{n-1} - 4x_{n-2} + y_{n} \]
	Ist die Folge $(y_n)_{n\geq 0}$ ebenfalls C-rekursiv, so genügt die
	Folge $(x_n)_{n\geq 0}$ einer homogenen Rekursion mit
	charakteristischem Polynom $\chi_x(z)\cdot\chi_y(z)$ (Skript S. 83).

	Das charakteristische Polynom von $(y_n)_{n\geq 0}$ erhält man
	beispielsweise über die Reihenent\-wicklung der $z$-Transformierten
	\[ y(z) = \sum_{n\geq 0} y_n z^n = \frac{c(z)}{d(z)}. \]
	Das Polynom $d(z)$ entspricht dabei dem \textit{Rekursionspolynom} zu
	$(y_n)_{n\geq 0}$, aus dem sich unmittelbar das charakteristische
	Polynom ableiten lässt.

	Die resultierende Rekursion entspricht genau der homogenen Rekursion,
	die in \textbf{2)} durch Elimination des inhomogenen Terms entstand.
	Nachdem die Eigenwerte dieser Rekursion bestimmt sind verfährt man wie
	bisher.
  \end{subparagraph}
\end{paragraph}
