%%%
% Angabe: Eine zweite Folge zur \textsc{Fibonacci}-Rekursion
%%%
Die \textsc{Fibonacci}-Folge $(F_n)_{n \geq 0} = (0,1,1,2,3,5,8,\ldots)$ (siehe Aufgabe: Rekursion und Induktion mit Fibonacci
und Abschnit 3.1.1 im Skriptum) genügt der linearen Rekursion
\begin{equation*}
x_{n} = x_{n-1} + x_{n-2}~~~(n \geq 2) \tag{$*$}
\end{equation*}
mit den Anfangswerten $F_0=0$ und $F_1=1$. Eine weitere bekannte Folge, die dieser Rekursion $(*)$
genügt, ist die \textsc{Lucas}-Folge\footnote{
Benannt nach den französischen Mathematiker Edouard \textsc{Lucas} (1842--1891), von dem
vemutlich auch die Bezeichnung der $F_n$ als \textsc{Fibonacci}-Zahlen stammt ---
fast 700 Jahre nach \textsc{Fibonacci}. 
\textsc{Lucas} hat sich viel mit Zahlentheorie befasst (Primzahltests!) und daneben mit seinen Büchern 
\emph{R\'ecr\'eations Math\'ematiques} (4 Bände) viel zur Popularisierung mathematischer
Denkweisen beigetragen. Von ihm (anonym veröffentlicht) stammen übrigens auch die 
bekannten \emph{Türme von Hanoi} und viele weitere populäre mathematische Probleme.}
\[
(L_n)_{n \geq 0} = (2,1,3,4,7,11,18,29, \ldots),
\]
die durch ihre Anfangswerte $L_0=2, L_1=1$ festgelegt ist.

\begin{flushenum}
\item Stellen Sie die Zahlen $L_n$ mit Hilfe von $\phi$ und $\widehat{\phi}$ dar (analog zur Aussage
	1. der Aufgabe: Rekursion und Induktion mit Fibonacci. Beachten Sie: die Folge $(L_n)_{n \geq 0}$ ist eine Linearkombination
	der beiden Folgen $(\phi^n)_{n \geq 0}$ und  $(\widehat{\phi}^n)_{n \geq 0}$.\\
	Notabene: $\widehat{\phi}^n$ ist als $\left( \widehat{\phi} \right)^n$ zu lesen.
\item Was ist $L_{n+1} L_{n-1} - L_n^2$? (analog  zur Aussage 2. der Aufgabe: Rekursion und Induktion mit Fibonacci
\item Was ist $\lim_{n \rightarrow \infty} \frac{L_{n+1}}{L_n}$ ~?
\item Wieviele Dezimalstellen hat die Zahl $L_n$ ~?
\item Die beiden Folgen $(F_n)_{n \geq 0}$ und $(L_n)_{n \geq 0}$ sind (offensichtlich!)
	linear-unabhängig im Vektorraum aller Folgen, die der Rekursion $(*)$ genügen. 
	Sie bilden also eine Basis dieses Raumes, ebenso, wie die beiden
	Folgen $(\phi^n)_{n \geq 0}$ und $(\widehat{\phi}^n)_{n \geq 0}$ eine Basis dieses
	Raumes bilden. Stellen Sie  $(\phi^n)_{n \geq 0}$ und $(\widehat{\phi}^n)_{n \geq 0}$
	als Linearkombinationen der beiden Folgen $(F_n)_{n \geq 0}$ und $(L_n)_{n \geq 0}$ dar.
\end{flushenum}


