\section*{Aufgabe 2 - Laufzeitenvergleich}
\addcontentsline{toc}{subsection}{Aufgabe 2 - Laufzeitenvergleich}
\begin{enumerate}[a)]
\item
	Offensichtlich ist das asymptotische Verhalten der Algorithmen folgendes:
	\[ t_\mathcal{A} \in \mathcal{O}(n) \]
	\[ t_\mathcal{B} \in \mathcal{O}(n \log n) \]
	\[ t_\mathcal{C} \in \mathcal{O}(n^2) \]
	Und es gilt für große $n$, dass $t_\mathcal{A} < t_\mathcal{B} < t_\mathcal{C}$.
	Für kleine $n$ muss man dies jedoch genauer analysieren:
	\lstset{language=Mathematica}
	\begin{lstlisting}
		tA[n_] = 4 n + 20;
		tB[n_] = n Log[2,n] + 2n + 4;
		tC[n_] = n (n-1) + 4;
		n/.Solve[tA[n] == tB[n], n]//N
		n/.Solve[tA[n] == tC[n], n]//N
		n/.Solve[tB[n] == tC[n], n]//N
		  {10.9815}  (* A == B *)
		  {-2.21699,7.21699}  (* A == C *)
		  {0.1375,5.44491}  (* B == C *)
	\end{lstlisting}
	Damit erhält man folgende Abschätzungen:
	\begin{itemize}
		\item $n=0$: $t_\mathcal{B} = t_\mathcal{C} < t_\mathcal{A}$
		\item $1 \leq n \leq 5$: $t_\mathcal{C} < t_\mathcal{B} < t_\mathcal{A}$
		\item $6 \leq n \leq 7$: $t_\mathcal{B} < t_\mathcal{C} < t_\mathcal{A}$
		\item $8 \leq n \leq 10$: $t_\mathcal{B} < t_\mathcal{A} < t_\mathcal{C}$
		\item $n > 10$: $t_\mathcal{A} < t_\mathcal{B} < t_\mathcal{C}$
	\end{itemize}

\item 
	\begin{enumerate}[1.]
	\item
		Bei dieser Aufgabe ist es sinnvoller von der Laufzeit der beiden Algorithmen den Logarithmus zu berechnen,
		da die Ausdrücke so leichter zu vergleichen sind. Da der Logarithmus streng monoton ist,
		sind die gefundenen Nullstellen und der Größenvergleich identisch mit dem der ursprünglichen Laufzeiten.
		\[ \log_2 t_\mathcal{A}(n) = \frac{1}{2} \log_2 n \]
		\[ \log_2 t_\mathcal{B}(n) = \sqrt{\log_2 n} \]
		$\mathcal{B}$ ist damit der schnellere Algorithmus, da:
		\[ \lim_{n \rightarrow \infty} \frac{\log_2 t_\mathcal{B}}{\log_2 t_\mathcal{A}} = 
		   \lim_{n \rightarrow \infty} \frac{\sqrt{\log_2 n}}{\frac{1}{2} \log_2 n} =
		   \lim_{n \rightarrow \infty} \frac{2}{\sqrt{\log_2 n}} = 0 \]
	\item
		\begin{lstlisting}
			tA[n_] = 1/2 Log[2,n];
			tB[n_] = Sqrt[Log[2,n]];
			n/.Solve[tA[n] == tB[n], n]
			  {1,16}
		\end{lstlisting}
		Für $n=1$ und $n=16$ sind beide Algorithmen gleich schnell, für $1 < n < 16$ ist $\mathcal{A}$ schneller,
		für $n > 16$ ist $\mathcal{B}$ schneller.
	
	\item
		Mit Teilaufgabe a) folgt $t_\mathcal{B} \in \mathcal{o}(t_\mathcal{A})$.
	\end{enumerate}
\end{enumerate}
