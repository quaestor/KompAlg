\section*{Aufgabe 20 - Aufwand für banale und schnelle Exponentiation im
Vergleich}
\addcontentsline{toc}{subsection}{Aufgabe 20 - Aufwand für banale und schnelle
Exponentiation im Vergleich}
\begin{enumerate}
\item
  Aus dem Algorithmus ergeben sich die rekursiven Folgen
  \[ y_{n}^{'} = (a^{2^{n}})_{n\geq 0} \]
  \[ x_{n}^{'} = x_{n-1}^{'} \cdot y_{n}^{'}, \quad x_{0}^{'} = a \]
  Betrachtet man nur die Exponenten so erhält man
  \[ y_{n} = 2y_{n-1} \]
  \[ x_{n} = x_{n-1} + y_{n} \]
  mit den Rekursionspolynomen
  \[ a_{y}(z) = 1 - 2z \]
  sowie
  \[ a_{x}(z) = 1 - z \]
  für den homogenen Teil von $(x_{n})$.
  Laut Skript $3.11$ (Kap. $3$) genügt die inhomogene Rekursion $(x_{n})$ einer
  homogenen Rekursion mit Rekursionspolynom
  \[ a_{x}(z) \cdot a_{y}(z) = (1 - z)(1 - 2z) = 1 - 3z + 2z^{2} \]
  und zugehörigem charakteristischen Polynom
  \[ \chi(z)= z^{2} - 3z + 2 = (z - 1)(z - 2) \]
  also den Eigenwerten $1$ und $2$; in Exponentialsummendarstellung:
  \[ x_{n} = \alpha_{1}1^{n} + \alpha_{2}2^{n} \]
  Mit den aus dem Algorithmus offensichtlichen Startwerten für die Exponenten
  ergeben sich folgende Werte für $\alpha_{1}$ und $\alpha_{2}$:
  \[ \begin{pmatrix} \alpha_{1} & \alpha_{2} \end{pmatrix} 
     \begin{pmatrix} 1 & 1 \\ 1 & 2 \end{pmatrix} = \begin{pmatrix} 0 & 1
     \end{pmatrix} \]
  \[ \alpha_{1} = -1, \quad \alpha_{2} = 1 \]
  \[ \Rightarrow x_{n} = 2^{n} - 1 \]
  Der Algorithmus ist also korrekt und berechnet $a^{2^{n}-1}$
\item \dots
\end{enumerate}
