\section*{Eingeschränkte Kanäle}
\addcontentsline{toc}{subsection}{Eingeschränkte Kanäle}
%%%
% Angabe
%%%
\subsection*{Angabe}
Das Szenario zu dieser Aufgabe ist im Problem 2.4 beschrieben.
\footnote{
Für alle drei hier behandelten (und ähnlich spezifizierte eingeschränkte
Sprachen) gibt es übrigens technische Motivationen und Anwendungen: bei der
Sprache $L$ etwa kann man sich unter einer ``1'' einen elektromagnetischen
Impuls vorstellen, unter einer ``0'' die Abwesenheit eines solchen. Das Verbot
von zwei aufeinanderfolgenden Einsen kann durch die Erholzeit des Impulsgebers
bedingt sein. Die Sprache $K$ kann Magnetisierungen beschreiben (keine
``Inseln''), die Sprache $M$ Synchronisation bei Rotation mit schwankender
Umdrehungszahl.
}

Es sei $\mathbb{B} = \{0,1\}$ das zweielementige Alphabet. Ein Wort $w \in
\mathbb{B}^+$ hat einen \emph{Faktor}~$v \in \mathbb{B}^+$, wenn es Wörter
$x,y \in \mathbb{B}^*$ gibt mit $w=x \cdot v \cdot y$

über dem Alphabet $\mathbb{B}$ werden die folgenden Sprachen betrachtet:
\begin{itemize}
\item[--] Die Sprache $L$ besteht aus allen Wörtern $w \in \mathbb{B}^*$, in
	denen $v=11$ nicht als Faktor vorkommt. Ein typisches Wort in $L$ ist
	beispielsweise $00100101001010100010$.

\item[--] Die Sprache $K$ besteht aus allen Wörtern $w \in \mathbb{B}^*$, in
	denen $u=010$ und $v=101$ nicht als Faktoren vorkommen.\\* Ein
	typisches Wort in $L$ ist beispielsweise $00111001110011000011$.

\item[--] Die Sprache $M$ besteht aus allen Wörtern $w \in \mathbb{B}^*$, in
	denen $u=000$ und $v=111$ nicht als Faktoren vorkommen.\\* Ein
	typisches Wort in $L$ ist beispielsweise $10010011011001101001$.
\end{itemize}

\begin{flushenum}
\item Geben Sie für jede der Sprachen $K,L,M$ einen regulären Ausdruck an.

\item Konstruieren Sie für jede der drei Sprachen $K,L,M$ einen endlichen
	deterministischen Automaten, der diese Sprache akzeptiert.

\item Verwenden Sie Trellis-Diagramme für diese Automaten, um herauszufinden,
	wieviele Wörter gegebener Länge $n$ jede dieser Sprachen enthält. Sei
	also
	$k_n = \sharp (K \cap \mathbb{B}^n)$,
	$\ell_n = \sharp (L \cap \mathbb{B}^n)$,
	$m_n = \sharp (M \cap \mathbb{B}^n)$,
	so bestimmen Sie $(k_n)_{0 \leq n \leq 8}$,
	$(\ell_n)_{0 \leq n \leq 8}$,
	$(m_n)_{0 \leq n \leq 8}$.

\item Vergleiche Sie die numerische Resultate. Was fällt Ihnen auf?  Haben Sie
	eine Erklärung dafür?

\item Geben Sie zu jedem der drei endlichen deterministischen Automaten die
	Transitionsmatrix an, die für jedes Paar $(i,j)$ von Zuständen angibt,
	wieviele der Inputsymbole $\in \mathbb{B}$ eine Transition von $i$ nach
	$j$ bewirken. Berechnen Sie das charakteristische Polynom zu jeder
	dieser Matrizen und bestimmen Sie die Eigenwerte, insbesondere den (dem
	Betrag nach) größten Eigenwert.\footnote{
	Für $L$ ist das einfach, denn man hat es mit einer $(2 \times
	2)$-Matrix zu tun.  Für $K$ und $M$ ist das weniger schwierig, als es
	zunächst den Anschein hat, auch wenn die entsprechenden Matrizen $(4
	\times 4)$-Matrizen sind. Das charakteristische Polynom vom Grad 4
	lässt sich in zwei Polynome vom Grad 2 faktorisieren -- und
	quadratische Gleichungen können Sie per Hand lösen. Sie können aber
	auch geeignete Software zur Hilfe nehmen.
	}
\end{flushenum}


%%%
% Lösung
%%%
\subsection*{Lösung}
\begin{flushenum}
\item
  \[ L = (0 | 10)^* (1 | \epsilon) \]
  \[ K = (0 | 1 | \epsilon) (0 (0^+) | 1 (1^+))^* (0 | 1 | \epsilon) \]
  \[ M = (\epsilon | 1 | 11) ((0 | 00) (1 | 11))^* (\epsilon | 0 | 00) \]

\item
  Die umrahmten Bereiche enthalten die Zustände, die für
  das asymptotische Wachstum relevant sind.

  \begin{tikzpicture}[shorten >=1pt,node distance=2cm,on grid,auto]
    % helper grid
    % \draw[help lines] (-1,-10) grid (10,2);
    % L
    \draw[fill=orange!25,rounded corners=5pt] (-0.8,-2.7) rectangle (0.8,1.5);
    \node[state,initial,initial text=L,accepting] (A)                 {$A$};
    \node[state,accepting]                (B)    [below=of A]         {$B$};
    \node[state]                          (C)    [above right=of B]   {$C$};
    \path[->] (A) edge [loop above] node        {0} ()
                  edge [bend left]  node        {1} (B)
              (B) edge [bend left]  node        {0} (A)
                  edge              node [swap] {1} (C)
              (C) edge [loop right] node        {0,1} ();
    % K
    \draw[fill=red!25,rounded corners=5pt] (4.7,-4.9) rectangle (8.1,1.0);
    \node[state,initial,initial text=K,accepting] (FK)   at (4,-2)    {$F$};
    \node[state,accepting]                (AK)   [above right=of FK]  {$A$};
    \node[state,accepting]                (CK)   [below right=of FK]  {$C$};
    \node[state,accepting]                (BK)   [right=of AK]        {$B$};
    \node[state,accepting]                (DK)   [right=of CK]        {$D$};
    \node[state]                          (EK)   [above right=of DK]  {$E$};
    \path[->] (FK) edge              node            {0} (AK)
                   edge              node [swap]     {1} (CK)
              (AK) edge [loop above] node            {0} ()
                   edge              node            {1} (BK)
              (BK) edge              node            {0} (EK)
                   edge              node [near end] {1} (CK)
              (CK) edge              node [swap]     {0} (DK)
                   edge [loop below] node            {1} ()
              (DK) edge              node [near end] {0} (AK)
                   edge              node [swap]     {1} (EK)
              (EK) edge [loop right] node            {0,1} ();
    % M
    \draw[fill=purple!25,rounded corners=5pt] (-0.3,-8) rectangle (3.1,-4);
    \node[state,initial,initial text=M,accepting] (AM) at (-1,-6)      {$A$};
    \node[state,accepting]                (BM) [above right=of AM]     {$B$};
    \node[state,accepting]                (CM) [right=of BM]           {$C$};
    \node[state,accepting]                (DM) [below right=of AM]     {$D$};
    \node[state,accepting]                (EM) [right=of DM]           {$E$};
    \node[state]                          (FM) [above right=of EM]     {$F$};
    \path[->] (AM) edge              node                   {0} (BM)
                   edge              node [swap]            {1} (DM)
              (BM) edge              node                   {0} (CM)
                   edge              node                   {1} (DM)
              (CM) edge              node                   {0} (FM)
                   edge              node [near start,swap] {1} (DM)
              (DM) edge [bend left]  node [swap]            {0} (BM)
                   edge              node                   {1} (EM)
              (EM) edge              node [near start,swap] {0} (BM)
                   edge              node [swap]            {1} (FM)
              (FM) edge [loop right] node                   {0,1} ();
  \end{tikzpicture}

\item Trellis-Diagramme:
\begin{itemize}
  \item $L$:

    \newcounter{x}
    \begin{tikzpicture}[place/.style={circle,draw=blue!50,fill=blue!20,thick,
                        inner sep=0pt,minimum size=6mm}]
      \node at (0.5,0) {$A$};
      \foreach \x/\xtext in {1/1,2/1,3/2,4/3,5/5,6/8,7/13,8/21,9/34}
        \node[xshift=\x*1.4 cm] (T-0-\x) [place] at (0,0) {$\xtext$};
      \node at (0.5,-1) {$B$};
      \foreach \x/\xtext in {1/,2/1,3/1,4/2,5/3,6/5,7/8,8/13,9/21}
        \node[xshift=\x*1.4 cm] (T-1-\x) [place] at (0,-1) {$\xtext$};
      \foreach \x in {1,...,8}
        \setcounter{x}{\x}\stepcounter{x}
        \path[->] (T-0-\x) edge (T-0-\arabic{x})
                           edge (T-1-\arabic{x});
      \foreach \x in {2,...,8}
        \setcounter{x}{\x}\stepcounter{x}
        \path[->] (T-1-\x) edge (T-0-\arabic{x});
      \node at (0.5,-1.7) {$n$};
      \node at (0.5,-2.2) {$l_n$};
      \foreach \x/\y in {0/1,1/2,2/3,3/5,4/8,5/13,6/21,7/34,8/55} {
        \node[xshift=\x*1.4 cm] at (1.4,-1.7) {$\x$};
        \node[xshift=\x*1.4 cm] at (1.4,-2.2) {$\y$};
      }
    \end{tikzpicture}
    \[ (l_n)_{0\leq n\leq8} = (1, 2, 3, 5, 8, 13, 21, 34, 55) \]
  \item $K$:

    \begin{tikzpicture}[place/.style={circle,draw=blue!50,fill=blue!20,thick,
                        inner sep=0pt,minimum size=6mm}]
      \node at (0.5,0) {$F$};
      \foreach \x/\xtext in {1/1,2/,3/,4/,5/,6/,7/,8/,9/}
        \node[xshift=\x*1.4 cm] (T-0-\x) [place] at (0,0) {$\xtext$};
      \node at (0.5,-1) {$A$};
      \foreach \x/\xtext in {1/,2/1,3/1,4/2,5/3,6/5,7/8,8/13,9/21}
        \node[xshift=\x*1.4 cm] (T-1-\x) [place] at (0,-1) {$\xtext$};
      \node at (0.5,-2) {$B$};
      \foreach \x/\xtext in {1/,2/,3/1,4/1,5/2,6/3,7/5,8/8,9/13}
        \node[xshift=\x*1.4 cm] (T-2-\x) [place] at (0,-2) {$\xtext$};
      \node at (0.5,-3) {$C$};
      \foreach \x/\xtext in {1/,2/1,3/1,4/2,5/3,6/5,7/8,8/13,9/21}
        \node[xshift=\x*1.4 cm] (T-3-\x) [place] at (0,-3) {$\xtext$};
      \node at (0.5,-4) {$D$};
      \foreach \x/\xtext in {1/,2/,3/1,4/1,5/2,6/3,7/5,8/8,9/13}
        \node[xshift=\x*1.4 cm] (T-4-\x) [place] at (0,-4) {$\xtext$};
      % connections
      \path[->] (T-0-1) edge (T-1-2) edge (T-3-2);
      \foreach \x in {2,...,8} {
        \setcounter{x}{\x}\stepcounter{x}
        \path[->] (T-1-\x) edge (T-1-\arabic{x})
                           edge (T-2-\arabic{x});
        \path[->] (T-3-\x) edge (T-3-\arabic{x})
                           edge (T-4-\arabic{x});
      }
      \foreach \x in {3,...,8} {
        \setcounter{x}{\x}\stepcounter{x}
        \path[->] (T-2-\x) edge (T-3-\arabic{x});
        \path[->] (T-4-\x) edge (T-1-\arabic{x});
      }
      \node at (0.5,-4.7) {$n$};
      \node at (0.5,-5.2) {$k_n$};
      \foreach \x/\y in {0/1,1/2,2/4,3/6,4/10,5/16,6/26,7/42,8/68} {
        \node[xshift=\x*1.4 cm] at (1.4,-4.7) {$\x$};
        \node[xshift=\x*1.4 cm] at (1.4,-5.2) {$\y$};
      }
    \end{tikzpicture}
    \[ (k_n)_{0\leq n\leq8} = (1, 2, 4, 6, 10, 16, 26, 42, 68) \]
  \item $M$:

    \begin{tikzpicture}[place/.style={circle,draw=blue!50,fill=blue!20,thick,
                        inner sep=0pt,minimum size=6mm}]
      \node at (0.5,0) {$A$};
      \foreach \x/\xtext in {1/1,2/,3/,4/,5/,6/,7/,8/,9/}
        \node[xshift=\x*1.4 cm] (T-0-\x) [place] at (0,0) {$\xtext$};
      \node at (0.5,-1) {$B$};
      \foreach \x/\xtext in {1/,2/1,3/1,4/2,5/3,6/5,7/8,8/13,9/21}
        \node[xshift=\x*1.4 cm] (T-1-\x) [place] at (0,-1) {$\xtext$};
      \node at (0.5,-2) {$C$};
      \foreach \x/\xtext in {1/,2/,3/1,4/1,5/2,6/3,7/5,8/8,9/13}
        \node[xshift=\x*1.4 cm] (T-2-\x) [place] at (0,-2) {$\xtext$};
      \node at (0.5,-3) {$D$};
      \foreach \x/\xtext in {1/,2/1,3/1,4/2,5/3,6/5,7/8,8/13,9/21}
        \node[xshift=\x*1.4 cm] (T-3-\x) [place] at (0,-3) {$\xtext$};
      \node at (0.5,-4) {$E$};
      \foreach \x/\xtext in {1/,2/,3/1,4/1,5/2,6/3,7/5,8/8,9/13}
        \node[xshift=\x*1.4 cm] (T-4-\x) [place] at (0,-4) {$\xtext$};
      % connections
      \path[->] (T-0-1) edge (T-1-2) edge (T-3-2);
      \foreach \x in {2,...,8} {
        \setcounter{x}{\x}\stepcounter{x}
        \path[->] (T-1-\x) edge (T-2-\arabic{x})
                           edge (T-3-\arabic{x});
        \path[->] (T-3-\x) edge (T-1-\arabic{x})
                           edge (T-4-\arabic{x});
      }
      \foreach \x in {3,...,8} {
        \setcounter{x}{\x}\stepcounter{x}
        \path[->] (T-2-\x) edge (T-3-\arabic{x});
        \path[->] (T-4-\x) edge (T-1-\arabic{x});
      }
      \node at (0.5,-4.7) {$n$};
      \node at (0.5,-5.2) {$m_n$};
      \foreach \x/\y in {0/1,1/2,2/4,3/6,4/10,5/16,6/26,7/42,8/68} {
        \node[xshift=\x*1.4 cm] at (1.4,-4.7) {$\x$};
        \node[xshift=\x*1.4 cm] at (1.4,-5.2) {$\y$};
      }
    \end{tikzpicture}
    \[ (m_n)_{0\leq n\leq8} = (1, 2, 4, 6, 10, 16, 26, 42, 68) \]
\end{itemize}

\item
  Die Folgen $(l_n), (k_n)$ und $(m_n)$ wachsen wie die Fibonacci-Zahlen.
  $(l_n)$ ist sogar genau die Fibonacci-Folge.

\item
  \[ L = \begin{bmatrix} 1 & 1 \\ 1 & 0 \end{bmatrix} \]
  \[ \chi_L(\lambda) = det(L - \lambda \cdot \mathds{1}) 
     = (1 - \lambda)(-\lambda) - 1 = \lambda^2 - \lambda - 1 \]
  Eigenwerte:
  \[ \chi_L(\lambda) = 0 \Rightarrow \lambda_1
     = \Phi, \quad \lambda_2 = \hat \Phi \]
     \vspace{2em}
  \[ K = \begin{bmatrix} 1 & 1 & 0 & 0 \\ 0 & 0 & 1 & 0 \\
                         0 & 0 & 1 & 1 \\ 1 & 0 & 0 & 0 \end{bmatrix} \]
  \[ \chi_K(\lambda) = det(K - \lambda \cdot \mathds{1})
     = \begin{vmatrix} 1-\lambda & 1 & 0 & 0 \\
                       0 & -\lambda & 1 & 0 \\
                       0 & 0 & 1-\lambda & 1 \\
                       1 & 0 & 0 & -\lambda \end{vmatrix} = \]
  \[ = -\lambda \cdot \begin{vmatrix} 1-\lambda & 0 & 0 \\ 
                                      0 & 1-\lambda & 1 \\
                                      1 & 0 & -\lambda \end{vmatrix} -
                      \begin{vmatrix} 1-\lambda & 1 & 0 \\
                                      0 & 0 & 1 \\
                                      1 & 0 & -\lambda \end{vmatrix} = \]
  \[ = (-\lambda)^2 (1-\lambda)^2 - 1 = \]
  \[ = \lambda^2 (1 - 2\lambda + \lambda^2) - 1 = \]
  \[ = \lambda^4 - 2\lambda^3 +\lambda^2 - 1 \]
  Eigenwerte:
  \[ \chi_K(\lambda) = 0 \Leftrightarrow (\lambda^2 - \lambda)^2 - 1 
     = 0 \Leftrightarrow \lambda^2 - \lambda \pm 1 = 0 \]
  \[ \lambda_1 = \frac{1 + i\sqrt{3}}{2}, \quad
     \lambda_2 = \frac{1 - i\sqrt{3}}{2} \]
  \[ \lambda_3 = \Phi, \quad \lambda_4 = \hat \Phi \]
     \vspace{2em}
  \[ M = \begin{bmatrix} 0 & 1 & 1 & 0 \\ 0 & 0 & 1 & 0 \\
                         1 & 0 & 0 & 1 \\ 1 & 0 & 0 & 0 \end{bmatrix} \]
  \[ \chi_M(\lambda) = det(M - \lambda \cdot \mathds{1})
     = \begin{vmatrix} -\lambda & 1 & 1 & 0 \\
                       0 & -\lambda & 1 & 0 \\
                       1 & 0 & -\lambda & 1 \\
                       1 & 0 & 0 & -\lambda \end{vmatrix} = \]
  \[ = -\begin{vmatrix} 1 & 1 & 0 \\ 
                        -\lambda & 1 & 0 \\
                        0 & -\lambda & 1 \end{vmatrix} - \lambda \cdot
        \begin{vmatrix} -\lambda & 1 & 1 \\
                        0 & -\lambda & 1 \\
                        1 & 0 & -\lambda \end{vmatrix} = \]
  \[ = -(1 + \lambda) - \lambda ((-\lambda)^3 + 1 - (-\lambda)) = \]
  \[ = - 1 - \lambda - \lambda ((-\lambda)^3 + 1 + \lambda) = \]
  \[ = \lambda^4 - \lambda^2 - 2\lambda - 1 \]
  Eigenwerte:
  \[ \chi_M(\lambda) = 0 \Leftrightarrow 
     ((\lambda + 1) - \lambda^2)((\lambda + 1) + \lambda^2) = 0
     \Leftrightarrow \pm \lambda^2 + \lambda + 1 = 0 \]
  \[ \lambda_1 = \frac{-1 + i\sqrt{3}}{2}, \quad
     \lambda_2 = \frac{-1 - i\sqrt{3}}{2} \]
  \[ \lambda_3 = \Phi, \quad \lambda_4 = \hat \Phi \]
  Der dem Betrag nach größte Eigenwert ist jeweils der goldene Schnitt $\Phi =
  \frac{1 + \sqrt{5}}{2}$.
\end{flushenum}
