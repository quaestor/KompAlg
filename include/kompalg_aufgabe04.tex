\section*{Fibonacci}
\addcontentsline{toc}{subsection}{Fibonacci}
\subsection*{Angabe}
%%%
% Angabe
%%%

Die durch die Rekursion
\[
F_0=0,~ F_1=1,~ F_{n+1}=F_n+F_{n-1}~(n \geq 1)
\]
definierte Folge
$
(F_n)_{n \geq 0} = (0,1,1,2,3,5,8,13,21,34,\ldots)
$
der {\sc Fibonacci}-Zahlen (siehe auch Abschnit 3.1.1 des Skriptums)
muss immer wieder als Ur-Beispiel einer rekursiven Definition herhalten.
Tatsächlich hat diese Zahlenfolge aber so interessante Eigenschaften und ist
so häufig vorkommend in Natur, Wissenschaft, Kunst, Architektur, etc., dass
man ihr ganze Bücher, eine Serie von Tagungen und sogar eine eigene Zeitschrift 
(\textsc {The Fibonacci Quarterly}) gewidmet hat.
Der tiefere Grund ist der Zusammenhang mit dem \textit{Goldenen Schnitt},
also der reellen Zahl
\[
\phi = \frac{1 + \sqrt{5}}{2} = 1.61803\ldots,
\]
die positive Lösung der quadratischen Gleichung
\[
X^2=X+1.
\]
Die andere (negative) Lösung $(1-\sqrt{5})/2 = -0.61803\ldots$ wird mit
$\widehat{\phi}$ bezeichnet.

Im Zusammenhang mit dem euklidischen Algorithmus spielen die
{\sc Fibonacci}-Zahlen auch eine bedeutsame Rolle -- mehr davon später.

Beweise über rekursive Programme bzw. rekursiv definierte Objekte
führt man mittels Induktion. Trainieren Sie hier Ihre Fertigkeiten in dieser Technik,

\begin{flushenum}
\item Beweisen Sie per Induktion
\[
F_n = \frac{\phi^n - (\widehat{\phi})^n}{\sqrt{5}}~~(n \geq 0) 
 \]
\item Beweisen Sie per Induktion
 \[
F_{n+1} F_{n-1} - F_n^2 = (-1)^n~~(n \geq 1)
\]
\item Beweisen Sie ebenso sorgf\"altig (im Sinne der Analysis), dass
\[
\lim_{n \rightarrow \infty} \frac{F_{n+1}}{F_n} = \phi
\] 
gilt. Veranschaulichen Sie grafisch das Konvergenzverhalten der Folge der Quotienten 
\[
\left(\frac{F_{n+1}}{F_n}\right)_{n \geq 1}
\]
d.h. tragen Sie die ersten numerischen Werte auf einer Zahlengeraden ab und beobachten Sie,
wie sich die Folge entwickelt.

\item
Wie gross sind $F_{100}, F_{200}, F_{300}, F_{400}$? Damit ist gemeint: wieviel
Dezimalziffern bzw. Bin\"arziffern braucht man, um diese Zahlen hinzuschreiben?
Sie sollen diese Zahlen nicht ausrechnen, sondern eine gute Sch\"atzung abgeben,
ohne viel zu rechnen!
\end{flushenum}




%%%
% Lösung
%%%
\subsection*{Lösung}
\begin{flushenum}
\item
  zu zeigen:
  \[ f_n = \frac{\Phi^n - \widehat \Phi^n}{\sqrt{5}} \quad n \geq 0 \]
  \begin{itemize}
    \item \textit{I.A.}:

      $n = 0$
        \[ f_0 = \frac{\Phi^0 - \widehat \Phi^0}{\sqrt{5}} 
               = \frac{1 - 1}{\sqrt{5}} = 0 \]
      $n = 1$
        \[ f_1 = \frac{\Phi - \widehat \Phi}{\sqrt{5}} 
               = \frac{1 + \sqrt{5} - 1 + \sqrt{5}}{2 \sqrt{5}} = 1 \]
    \item \textit{I.V.}: Die Behauptung gelte für $n$.
    \item \textit{I.S}: $n \rightarrow n + 1$

    Da $\Phi$ und $\widehat \Phi$ die beiden Lösungen der quadratischen Gleichung
    $x^2 - x - 1 = 0$ sind, gilt offensichtlich $\Phi + 1 = \Phi^2$ sowie
    $\widehat \Phi + 1 = \widehat \Phi^2$.
%    \[ \Phi + 1 = \frac{3 + \sqrt{5}}{2} = \frac{6 + 2 \sqrt{5}}{4} 
%                = \frac{1 + s \sqrt{5} + 5}{4} = (\frac{1 + \sqrt{5}}{2})^2 
%                = \Phi^2 \]
%    analog gilt:
%    \[ \widehat \Phi + 1 = \widehat \Phi^2 \]
    
    Damit folgt:
    \[ f_{n+1} = f_n + f_{n-1} \overset{IV}{=} \frac{\Phi^n - \widehat
       \Phi^n}{\sqrt{5}} + \frac{\Phi^{n-1} - \widehat \Phi^{n-1}}{\sqrt{5}} = \]
    \[ = \frac{(\Phi^n + \Phi^{n-1}) - (\widehat \Phi^n + \widehat \Phi^{n-1})}{\sqrt{5}}
       = \frac{(\Phi^{n-1} \cdot \Phi + \Phi^{n-1}) - 
         (\widehat \Phi^{n-1} \cdot \widehat \Phi + \widehat \Phi^{n-1})}{\sqrt{5}} = \]
    \[ = \frac{(\Phi^{n-1} (\Phi + 1)) - (\widehat \Phi^{n-1} (\widehat \Phi + 1))}{\sqrt{5}} 
       = \frac{\Phi^{n-1} \Phi^2 - \widehat \Phi^{n-1} \widehat \Phi^2}{\sqrt{5}} = \]
    \[ = \frac{\Phi^{n+1} - \widehat \Phi^{n+1}}{\sqrt{5}} \]
    $\hfill \square$
  \end{itemize}

\item
zu zeigen:
  \[ f_{n+1} f_{n-1} - f_n^2 = (-1)^n \quad n \geq 1 \]
  \begin{itemize}
    \item \textit{I.A.}: $n = 1$
      \[ f_2f_0 - f_1^2 = 1 \cdot 0 - 1^2 = (-1)^1 \]
    \item \textit{I.V.}: Die Annahme gelte für $n$.
    \item \textit{I.S.}: $n \rightarrow n + 1$

    Mit dem Zusammenhang $f_{n} = f_{n-1} + f_{n-2}$ für alle $n \geq 2$ folgt:
    \[ f_{n+2}f_n - f_{n+1}^2 = (f_{n+1} + f_n)(f_{n+1} - f_{n-1}) - f_{n+1}^2
                              = -f_{n+1}f_{n-1} + f_{n+1}f_n - f_nf_{n-1} = \]
    \[ = -1(f_{n+1}f_{n-1} - (f_{n+1}f_n - f_nf_{n-1})) 
       = -1(f_{n+1}f_{n-1} - (f_n(f_{n+1} - f_{n-1}))) = \]
    \[ = -1(f_{n+1}f_{n-1} - f_n^2) \overset{IV}{=} -1(-1)^n = (-1)^{n+1} \]
    $\hfill \square$
  \end{itemize}
\item
  zu zeigen:
  \[ \lim_{n \rightarrow \infty} \frac{f_{n+1}}{f_n} = \Phi \]
  Vorbemerkung:
  \[ \lim_{n \rightarrow \infty} (\overbrace{\frac{\widehat \Phi}{\Phi}}^{< 1})^n = 0 \]
  Beweis:
  \[ \lim_{n \rightarrow \infty} \frac{f_{n+1}}{f_n} 
     = \lim_{n \rightarrow \infty} \frac{\Phi^{n+1} - 
       \widehat \Phi^{n+1}}{\Phi^n - \widehat \Phi^n} = \]
  \[ = \lim_{n \rightarrow \infty} \frac{\Phi^{n+1}}{\Phi^n - \widehat \Phi^n} - 
       \lim_{n \rightarrow \infty} \frac{\widehat \Phi^{n+1}}{\Phi^n - \widehat \Phi^n} = \]
  \[ = \lim_{n \rightarrow \infty} \frac{\Phi^n \cdot \Phi}{\Phi^n (1 - \frac{\widehat \Phi^n}{\Phi^n})} - 
       \lim_{n \rightarrow \infty} \frac{\frac{\widehat \Phi^{n+1}}{\Phi^{n+1}}}
                                        {\Phi^{n+1}(\Phi^{-1} - \frac{\widehat \Phi^n}{\Phi^{n+1}})} = \]
  \[ = \lim_{n \rightarrow \infty} \frac{\Phi}{1 - (\frac{\widehat \Phi}{\Phi})^n} - 
       \lim_{n \rightarrow \infty} \frac{(\frac{\widehat \Phi}{\Phi})^{n+1}}
                                        {\Phi^{-1} (1 - (\frac{\widehat \Phi}{\Phi})^n)} 
     = \Phi - 0 = \Phi \] $\hfill \square$
  \begin{table}[h!b!p!]
  \caption{Wertetabelle:}
  \begin{center}
  \begin{tabular}{r|cccccccccc}
  $n$                   & 1  &  2  &  3  &  4  &  5  &  6  &  7  &  8  &  9  & 10 \\
  \hline
  $\frac{f_{n+1}}{f_n}$ & 1  &  2  & $\frac{3}{2}$ & $\frac{5}{3}$ & $\frac{8}{5}$ & $\frac{13}{8}$ & $\frac{21}{13}$ & $\frac{34}{21}$ & $\frac{55}{34}$ & $\frac{89}{55}$
  \end{tabular}
  \end{center}
  \end{table}

  \parindent 0pt
  Zahlengerade: \\[0.5em]
  \begin{tikzpicture}[scale=1,cap=round]
  \tikzstyle{axes}=[]
  \begin{scope}[style=axes]
    \draw[loosely dotted] (-2,0) -- (-1,0);
    \draw (-1,0) -- (7,0);
    \draw[loosely dotted] (7,0) -- (9,0);
    \draw[->] (9,0) -- (12,0);
    \foreach \x/\xtext/\xname in {0.0/1.5/A, 6.4/1.66/C, 4.76/1.615/E}
      \draw[xshift=\x cm] (0pt,2pt) -- (0pt,-2pt) node[above=3pt,fill=white] (\xname) {$\xtext$};
    \foreach \x/\xtext/\xname in {4.0/1.6/B, 5.00/1.625/D, 10.0/2.0/F}
      \draw[xshift=\x cm] (0pt,2pt) -- (0pt,-2pt) node[below] (\xname) {$\xtext$};
    \draw[->] (F.south) to [out=270,in=270] ($(A.south) +(0.0,-5pt)$);
    \draw[->] (A.north) to [out=90,in=90] (C.north);
    \draw[->] ($(C.south) +(0.0,-5pt)$) to [out=270,in=270] (B.south);
    \draw[->] ($(B.north) +(0.0,5pt)$) to [out=90,in=90] ($(D.north) +(0.0,15pt)$);
    \draw[->] (D.south) to [out=270,in=300] ($(E.south) +(0.0,-15pt)$);
  \end{scope}
  \end{tikzpicture}

\item
$f_n$ wächst wie $\frac{\Phi^n}{\sqrt{5}}$ (siehe Vorlesung)
\[ \Rightarrow \frac{\Phi^n}{\sqrt{5}} = b^x \] wobei $b$ die Basis und $x$ die Anzahl der Stellen ist.
\[ x = \log_b \frac{\Phi^n}{\sqrt{5}} = n \log_b \Phi - \log_b \sqrt{5} \]
\begin{itemize}
  \item $\log_{10} \Phi \approx 0.20898,\quad \log_{10} \sqrt{5} \approx 0.34949\quad \Rightarrow 0.209\cdot n - 0.35 \textnormal{ Dezimalziffern}$
  \item $\log_{2}  \Phi \approx 0.69424,\quad \log_{2} \sqrt{5} \approx 1.16096\quad \Rightarrow 0.694\cdot n - 1.161 \textnormal{ Binärziffern}$
\end{itemize}
\begin{table}[h!b!p!]
\begin{center}
\begin{tabular}{r|cccc}
& $f_{100}$ & $f_{200}$ & $f_{300}$ & $f_{400}$ \\
\hline
Dezimalziffern &  21 & 42 & 63 & 84 \\
Binärziffern   &  69 & 138& 208& 277
\end{tabular}
\end{center}
\end{table}
\end{flushenum}
