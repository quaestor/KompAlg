\section*{Aufgabe 16 - Devil's Staircase}
\addcontentsline{toc}{subsection}{Aufgabe 16 - Devil's Staircase}
\begin{enumerate}
% teil 1: automaten interpretation, Ak für k = 2,3,4
\item
	\begin{itemize}
		\item $k= 2$: 
			\[ A2 = \begin{bmatrix}
					1 & 1 \\
					2 & 0 \\
				\end{bmatrix} \]
			\[ \chi_2(z) = 
				\begin{vmatrix}
					z - 1 & -1 \\
					-2    & z  \\
				\end{vmatrix} =
				z^2 - z - 2 = (z-2)(z+1) \]
			Die Eigenwerte sind $\lambda_1 = 2$ und $\lambda_2 = -1$.
		\item $k = 3$:
			\[ A3 = \begin{bmatrix}
					1 & 1 & 0 \\
					1 & 0 & 1 \\
					2 & 0 & 0 \\
				\end{bmatrix} \]
			\[ \chi_3(z) = 
				\begin{vmatrix}
					z-1 & -1 & 0 \\
					-1 & z & -1 \\
					-2 & 0 & z \\
				\end{vmatrix} =
				z^3 - z^2 - z - 2 = (z-2)(z^2 + z + 1) \]
			Die Eigenwerte sind $\lambda_1 = 2, \lambda_2 = \omega_3, \lambda_3 = \omega_3^2 $,
			wobei $\omega_3$ die dritte Einheitswurzel $e^{\imath 2 \pi / 3}$ ist.
		\item $k = 4$:
			\[ A4 =	\begin{bmatrix}
					1 & 1 & 0 & 0 \\
					1 & 0 & 1 & 0 \\
					1 & 0 & 0 & 1 \\
					2 & 0 & 0 & 0 \\
				\end{bmatrix} \]
			\[ \chi_4(z) = 
				\begin{vmatrix}
					z-1 & -1 & 0 & 0 \\
					-1 & z & -1 & 0 \\
					-1 & 0 & z & -1 \\
					-2 & 0 & 0 & z \\
				\end{vmatrix} = 
				(z-1) z^3 + 
					\begin{vmatrix}
						-1 & -1 & 0 \\
						-1 & z & -1 \\
						-2 & 0 & z \\
					\end{vmatrix} = \] \[
				(z-1)z^3 - z^2 - z  - 2 =
				z^4 - z^3 - z^2 - z - 2 =
				(z-2)(z^3 + z^2 + z + 1) \]
			Die Eigenwerte sind $\lambda_1 = 2, \lambda_2 = \omega_4, \lambda_3 = \omega_4^2, \lambda_4 = \omega_4^3$,
			wobei $\omega_4$ die vierte Einheitswurzel $e^{\imath 2 \pi / 4} = \imath$ ist.
	\end{itemize}

% teil 2: allgemeine Ak
\item Allgemein sieht die Matrix $Ak$ folgendermaßen aus:
	\[ Ak = \begin{bmatrix}
			1      & 1      & 0      & \cdots & \cdots & \cdots & 0      \\
			1      & 0      & 1      & 0      &        &        & \vdots \\
			\vdots & \vdots & \ddots & \ddots & \ddots &        & \vdots \\
			\vdots & \vdots &        & \ddots & \ddots & \ddots & \vdots \\
			\vdots & \vdots &        &        & \ddots & \ddots & 0      \\
			1      & \vdots &        &        &        & \ddots & 1      \\
			2      & 0      & \cdots & \cdots & \cdots & \cdots & 0      \\
		\end{bmatrix} \]
	Das Charakteristische Polynom lässt sich rekursiv berechnen (Entwicklung nach erster Zeile,
	nachdem 2. Spalte auf 1. Spalte addiert wurde):
	\begin{eqnarray*}
		\chi_k(z) &=&
				\begin{vmatrix}
					z-1    & -1     & 0      & \cdots & \cdots & \cdots & 0      \\
					-1     & z      & -1     & 0      &        &        & \vdots \\
					\vdots & 0      & \ddots & \ddots & \ddots &        & \vdots \\
					\vdots & \vdots & \ddots & \ddots & \ddots & \ddots & \vdots \\
					\vdots & \vdots &        & \ddots & \ddots & \ddots & 0      \\
					-1     & \vdots &        &        & \ddots & \ddots & -1     \\
					-2     & 0      & \cdots & \cdots & \cdots & 0      & z      \\
				\end{vmatrix} = \\
			&=&
				\begin{vmatrix}
					z-2    & -1     & 0      & \cdots & \cdots & \cdots & 0      \\
					z-1    & z      & -1     & 0      &        &        & \vdots \\
					\vdots & 0      & \ddots & \ddots & \ddots &        & \vdots \\
					\vdots & \vdots & \ddots & \ddots & \ddots & \ddots & \vdots \\
					\vdots & \vdots &        & \ddots & \ddots & \ddots & 0      \\
					-1     & \vdots &        &        & \ddots & \ddots & -1     \\
					-2     & 0      & \cdots & \cdots & \cdots & 0      & z      \\
				\end{vmatrix} = \\
			&=& (z-2) \cdot
				\begin{vmatrix}
					& z      & -1     & 0      & \cdots & \cdots & 0      \\
                                        & 0      & \ddots & \ddots & \ddots &        & \vdots \\
                                        & \vdots & \ddots & \ddots & \ddots & \ddots & \vdots \\
                                        & \vdots &        & \ddots & \ddots & \ddots & 0      \\
                                        & \vdots &        &        & \ddots & \ddots & -1     \\
                                        & 0      & \cdots & \cdots & \cdots & 0      & z      \\
				\end{vmatrix} + \\
				& & \quad \quad \quad
				\begin{vmatrix}
					z-1    & -1     & 0      & \cdots & \cdots & 0      \\
					-1     & z      & -1     & \ddots &        & \vdots \\
					\vdots & 0      & \ddots & \ddots & \ddots & \vdots \\
					\vdots & \vdots & \ddots & \ddots & \ddots & 0      \\
					-1     & \vdots &        & \ddots & \ddots & -1     \\
					-2     & 0      & \cdots & \cdots & 0      & z      \\
				\end{vmatrix} = \\
			&=& (z-2) z^{k-1} + \chi_{k-1}(z)
	\end{eqnarray*}
	Mit Hilfe dieses Zusammenhangs kann man per vollständiger Induktion die gewünschte Formel beweisen:
	\begin{itemize}
		\item \textit{I.A.}: $k=2$: \[ \chi_2(z) = (z-2)(z+1) = (z-2)\frac{(z+1)(z-1)}{z-1} = (z-2)\frac{z^2 - 1}{z-1} \]
		\item \textit{I.V.}: Für ein $k \in \mathbb{N}$ gelte:
			\[ \chi_k(z) = (z-2)\frac{z^k - 1}{z-1} \]
		\item \textit{I.S.}: $k \rightarrow k+1$:
			\begin{eqnarray*}
			\chi_{k+1}(z) &=& (z-2) z^{k} + \chi_k(z) \overset{I.V.}{=} (z-2) z^k + (z-2) \frac{z^k - 1}{z-1} \\
			   &=& (z-2) \frac{z^k (z-1)}{z-1} + (z-2)\frac{z^k - 1}{z-1} =
			   (z-2) \frac{z^{k+1} - 1}{z-1}
			\end{eqnarray*}
	\end{itemize}
	Die Eigenwerte sind zum einen $\lambda_1 = 2$ und zum anderen die Lösungen von $z^k - 1 = 0$ (außer $z = 1$), also den
	$k$-ten Einheitswurzeln, $\lambda_i = e^{\imath 2 \pi \frac{j}{k}}$ mit $1 \leq j \leq k-1$, sie liegen
	auf einem Kreis mit Radius $1$ um den Ursprung.

\item 
Da die Folgen $\left( Ak_{1,j}^{(n)} \right)_{n \geq 0}$ C-Rekursionen sind, die das charakteristische Polynom $\chi_k(z)$
aus der vorigen Aufgabe haben ist der dominante Eigenwert $\lambda = \lambda_1 = 2$.
Aus der Problembeschreibung ergibt sich 
\[ Ak_{1,j}^{(n)} = Ak_{1,j-1}^{(n-1)} \quad \quad (1 < j \leq k) \]
da der Betrunkene erst auf der Stufe $j-1$ nach $n-1$ Schritten gewesen sein muss, wenn er Stufe $j$ nach $n$ Schritten erreichen will.
Auf Grund der Asymptotik ergibt sich somit:
\[ Ak_{1,j}^{(n)} = \alpha_j \cdot 2^n = Ak_{1,j-1}^{(n-1)} = \alpha_{j-1} \cdot 2^{n-1} \quad \quad (1 < j \leq k) \]
also $2 \cdot \alpha_j = \alpha_{j-1}$.

\[ Ak_{1,1}^{(n)} + Ak_{1,2}^{(n)} + \ldots + Ak_{1,k}^{(n)} = \alpha_1 \cdot 2^n + \alpha_2 \cdot 2^n + \ldots + \alpha_k \cdot 2^n = 2^n \]
ist die Summe aller Pfade der Länge $n$ in dem Graphen, die bei $1$ beginnen und auf einer beliebigen Stufe zwischen $1$ und $k$ enden.
Da der Graph einen Automaten beschreibt, der in jedem Zustand zwei Eingabewerte ($p$ und $q$) akzeptiert, sind dies offensichtlich
$2^n$ mögliche Pfade.
Somit folgt (*) $\alpha_1 + \alpha_2 + \ldots + \alpha_k = 1$.
Mit $2 \cdot \alpha_j = \alpha_{j-1} $ erhält man weiterhin 
\[ \alpha_1 + \alpha_2 + \ldots + \alpha_k = \alpha_1 \sum_{j=0}^{k-1} \left( \frac{1}{2}\right)^j 
   = \alpha_1  \frac{1 - \left(\frac{1}{2}\right)^k}{1 - \frac{1}{2}}  = 1 \]
Folglich ist 
\[ \alpha_1 = \frac{1}{\left(\frac{1}{2}\right)^{k-1} - 2} = \frac{2^{k-1}}{2^k - 1} \]
\[ \alpha_j = \left(\frac{1}{2}\right)^{j-1} \cdot \alpha_1 = \frac{2^{k-j}}{2^k - 1}\]

\item
  $A_{k,p}$ für $k = 2, 3, 4$:

  \begin{minipage}{0.4\linewidth}
  \[ A_{2,p} = \begin{bmatrix} 1-p & p \\ 1 & 0 \end{bmatrix} \]
  \[ \chi_{2,p}(z) = \begin{vmatrix} 1-p-z & p \\ 1 & -z \end{vmatrix} \]
  \[ = z^{2} + pz - z - p \]
  \[ \lambda_{1} = 1, \quad \lambda_{2}= -p \]
  \vspace{0.5em}
  \[ A_{3,p} = \begin{bmatrix} 1-p & p & 0 \\
                               1-p & 0 & p \\
                                 1 & 0 & 0 \end{bmatrix} \]
  \[ \chi_{3,p}(z) = \begin{vmatrix} 1-p-z & p  & 0 \\
                                       1-p   & -z & p \\
                                       1     & 0  & -z \end{vmatrix} \]
  \[ = -z^{3} - pz^{2} + z^{2} - p^{2}z + pz + p^{2} = \]
  \[ = -(z - 1) (z^{2} + pz + p^{2}) \]
  \[ \lambda_{1}= 1 \]
  \[ \lambda_{2/3}= -\frac{p}{2}(1 \pm i\sqrt{3}) \]
  \end{minipage}\hfill\begin{minipage}{0.5\linewidth}
  \[ A_{4,p} = \begin{bmatrix} 1-p & p & 0 & 0 \\
                               1-p & 0 & p & 0 \\
                               1-p & 0 & 0 & p \\
                                 1 & 0 & 0 & 0 \end{bmatrix} \]
  \[ \chi_{4,p}(z) = det(A_{4,p} - z\cdot \mathds{1}) = \]
  \[                 -\begin{vmatrix} p & 0 & 0 \\
                                     -z & p & 0 \\
                                      0 &-z & p \end{vmatrix} 
                     -\begin{vmatrix} 1-p-z & p & 0 \\
                                      1-p & -z & p \\
                                      1-p & 0 & -z \end{vmatrix} \]
  \[ = z^4 + pz^3 - z^3 + p^{2}z^2 - pz^2 + p^{3}z - p^{2}z -p^3 \]
  \[ = (z - 1) (z^{3} + pz^{2} + p^{2}z + p^{3}) \]
  \[ \lambda_{1}= 1, \]
  \[ \lambda_{2}= -p, \quad
     \lambda_{3}= ip, \quad
     \lambda_{4}= -ip\]
  \end{minipage}
  
\item
  \[ A_{k,p} = \left[a_{ij}\right]_{1\leq i,j\leq k}, \]
  \[ a_{ij} = \begin{cases} 1-p &\text{wenn~} j = 0, i \not= k \\
                              1 &\text{wenn~} j = 0, i = k \\
                              p &\text{wenn~} j = i + 1 \\
                              0 &\text{sonst} \end{cases} \]

  zu zeigen: \[ \chi_{k,p}(z) = (z - 1) \frac{z^{k}-p^{k}}{z-p}(-1)^{k} \]
  Beweis durch Induktion über $k$.
  \begin{itemize}
    \item Induktionsanfang: $k = 2$
      \[ \chi_{z,p}(z) = (z - 1) \frac{z^{2}-p^{2}}{z-p} = (z-1)(z+p) \]
    \item Induktionsvoraussetzung: Die Behauptung gelte für $k$.
    \item Induktionsschritt: $k \rightarrow k + 1$

      Addition der zweiten auf die erste Spalte:
      \begin{equation*}
      \begin{split} \chi_{A_{k}}(z) & = \begin{vmatrix}
                            1-p-z & p  & 0  & \cdots & \cdots & 0 \\
                            1-p & -z & p  & \ddots & \ddots & \vdots \\
                            \vdots & 0  & \ddots & \ddots & \ddots & \vdots \\
                            \vdots & \vdots  & \ddots & \ddots & \ddots & 0 \\
                            1-p & \vdots  & \ddots & \ddots & \ddots & p \\
                            1 & 0 & \cdots & \cdots & 0 & -z
                           \end{vmatrix} \\
                       & = \begin{vmatrix}
                            1-z & p  & 0  & \cdots & \cdots & 0 \\
                            1-p-z & -z & p  & \ddots & \ddots &\vdots \\
                            1-p & 0  & \ddots & \ddots & \ddots & \vdots \\
                            \vdots & \vdots  & \ddots & \ddots & \ddots & 0 \\
                            1-p & \vdots  & \ddots & \ddots & \ddots & p \\
                            1 & 0 & \cdots & \cdots & 0 & -z
                           \end{vmatrix} = \\
       & = -(z - 1) (-z)^{k-1} - p\chi_{A_{k-1}}(z) = \\
       & = (-1)^{k}(z - 1) z^{k-1} - p(z-1)\frac{z^{k-1}-p^{k-1}}{z-p} = \\
       & = (-1)^{k}(z - 1) \left( z^{k-1} + \frac{pz^{k-1}-p^{k}}{z-p} \right) = \\
       & = (-1)^{k}(z - 1) \left( \frac{z^{k}-z^{k-1}p}{z-p} +
                          \frac{pz^{k-1}-p^{k}}{z-p} \right) = \\
       & = (-1)^{k}(z - 1) \left( \frac{z^{k}-p^{k}}{z-p} \right)
       \end{split}
       \end{equation*}

      $\chi_{A_{k,p}}$ hat die Nullstellen $1$ und die $k$-ten Wurzeln von
      $p^{k}$, welche in der komplexen Ebene auf einem Kreis mit Radius $p$ um
      den Ursprung liegen.
  \end{itemize}

\item
  Da man die Stufe $S_{l}$ nur auf direktem Weg in $l - 1$ Schritten erreichen
  kann, muss man, um beim $n$-ten Schritt auf Stufe $S_{l}$ zu landen, zuvor
  nach $n - l - 1$ Schritten auf der ersten Stufe gestanden haben.

  Die Wahrscheinlichkeit ergibt sich aus dem Produkt der
  Einzelwahrscheinlichkeiten.

  \item
   \[ \lim_{n\rightarrow\infty} P_{1}^{(n)} = a, \quad
      1 = \sum_{i=1}^{k} a \cdot p^{l-1} = \sum_{i=0}^{k-1} a \cdot p^{l} = a
   \sum_{l=0}^{k-1} p^{l} = a \frac{1-p^{k}}{1-p} \]
  \[ \Rightarrow a = \frac{1-p}{1-p^{k}} \]

  \[ (\pi_{i})_{1\leq l\leq k} = (a\cdot p^{l-1})_{1\leq l\leq k} 
     = (\frac{p^{l-1} - p^{l}}{1-p^{k}})_{1\leq l\leq k} \]
\end{enumerate}
