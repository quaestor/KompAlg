%%%
% Angabe: Asymptotisches Verhalten von C-rekursiven Folgen
%%%
Ist $\boldsymbol{a}=(a_1,a_2, \ldots, a_k)$ und untersucht man Folgen
$\boldsymbol{x}= (x_n)_{n \geq 0}$, die der  Rekursion
\begin{equation*}
x_n = a_1 \, x_{n-1} + a_2  \, x_{n-2} + \cdots + a_k  \, x_{n-k}~~~(n \geq k) \tag{$*$}
\end{equation*}
genügen, so lassen sich diese Folgen als Linearkombinationen von geometrischen
Folgen darstellen, die zu den Nullstellen des charakteristischen Polynoms
\[
\chi_{\boldsymbol{a}}(z) = z^k -a_1  \, z^{k-1} - a_2  \, z^{k-2} - \cdots - a_k
\]
gehören. Ist
\[
\chi_{\boldsymbol{a}}(z)  = (z-\lambda_1)^{t_1}(z-\lambda_2)^{t_2} \cdots (z-\lambda_\ell)^{t_\ell}
\]
mit paarweise verschiedenen $\lambda_j~(1 \leq j \leq \ell)$, d.h. Nullstelle $\lambda_j$ hat die
Vielfachheit $t_j$, so gilt für jede Folge $\boldsymbol{x}= (x_n)_{n \geq 0}$, die $(*)$ erfüllt,
\[
x_n = p_1(n)\cdot \lambda_1^n + p_2(n)\cdot \lambda_2^{n} + \cdots + p_\ell(n) \cdot \lambda_\ell^{n}~~~(n \geq 0),
\]
wobei jedes $p_j(n)$ ein Polynom in $n$ vom Grad $t_j-1$ ist. Diese Polynome sind  durch die 
Anfangswerte $x_0,x_1, \ldots ,x_{k-1}$ eindeutig bestimmt.

Betrachten Sie nun die folgenden vier Fälle von Rekursionen der Ordnung 3:
\begin{flushalpha}
\item Die Rekursion $x_n = 2 \, x_{n-1} + 5 \, x_{n-2} - 6 \, x_{n-3}$.
\item Die Rekursion $x_n = 7 \, x_{n-1} - 16 \, x_{n-2} +12 \, x_{n-3}$.
\item Die Rekursion $x_n = 6 \, x_{n-1} - 12 \, x_{n-2} +8 \, x_{n-3}$.
\item Die Rekursion $x_n = 3 \, x_{n-1} - 3 \, x_{n-2} + \, x_{n-3}$.
\end{flushalpha}

Für diese vier Fälle sollen Sie folgendes tun:
\begin{flushenum}
\item Bestimmen Sie die Nullstellen des zugehörigen charakteristischen Polynoms und
	deren Vielfachheiten.\footnote{
	Die Beispiele sind so gewählt, dass die Nullstellen (kleine) ganze Zahlen sind, die
	Sie notfalls durch Ausprobieren ermitteln können. Sie können auch erst mal die
	Funktionsgraphen der charakteristischen Polynome plotten und inspizieren, wo sich
	die Nullstellen befinden.}
\item Wie lassen sich demnach die Folgen $\boldsymbol{x}= (x_n)_{n \geq 0}$, 
	die dieser Rekursion genügen, als
	Linearkombination von geometrischen Folgen darstellen?
\item Welche Möglichkeiten für das asymptotische Verhalten ergeben sich daraus
	(in Abhängigkeit von den Anfangswerten)?
\item Geben Sie für jede der Möglichkeit in 3. Anfangswerte an, für die diese
	Möglichkeit realisiert wird.
\end{flushenum}
Hinweis: alle Rekursionen haben die Ordnung $k=3$, es gibt also jeweils drei
	unterschiedlich schnell wachsende Beiträge. Somit gibt es in jeder Situation
	drei Möglichkeiten.


