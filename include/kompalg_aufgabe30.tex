\section*{Nochmal Sortieren}
\addcontentsline{toc}{subsection}{Nochmal Sortieren}



%%%
% Angabe
%%%
\subsection*{Angabe}
Ein Permutation $\sigma=\sigma_1\,\sigma_2\, \sigma_3\,\ldots \,\sigma_{2n}$ der Zahlen
$\{1,2,\ldots,2n\}$ soll \emph{perfekt} heissen, wenn folgendes gilt:
\[
\sigma_1 < \sigma_3 < \sigma_5 < \cdots < \sigma_{2n-1}~~~\text{und}~~~
\sigma_1 < \sigma_2, \, \sigma_3 < \sigma_4 , \,   \ldots , \, \sigma_{2n-1} < \sigma_{2n}
\]
Anschaulich bedeutet das: ordnet man die $2n$ Zahlen $\sigma_1, \ldots,\sigma_{2n}$
folgenderma\ss en spaltenweise in einer Matrix an
\[
\begin{bmatrix}
\sigma_1 & \sigma_3 & \ldots & \sigma_{2n-1} \cr
\sigma_2 & \sigma_4 & \ldots & \sigma_{2n}
\end{bmatrix}
\] 
so sind alle Spalten und die die erste Zeile dieser Matrix monoton wachsende Folgen.
Die zweite Zeile muss nicht monoton wachsend sein.

Als Beispiel: für $n=2$ gibt es drei perfekte Permutationen:
\[
1234 \leftrightarrow \begin{bmatrix} 1 & 3 \cr 2 & 4 \end{bmatrix} ~~~
1324\leftrightarrow \begin{bmatrix} 1 & 2 \cr 3 & 4 \end{bmatrix} ~~~~
 1423\leftrightarrow\begin{bmatrix} 1 & 2 \cr 4 & 3 \end{bmatrix} 
\]

\begin{flushenum}
\item Stellen Sie durch Auflisten aller Möglichkeiten fest, dass es genau 15 perfekte Permutationen 
	von $\{1,2,3,4,5,6\}$ gibt.
\item Bezeichnet nun $p(n)$ die Anzahl der perfekten Permutationen von $\{1,2,\ldots,2n\}$,
	so zeigen Sie die Rekursion $p(n)= (2n-1)\cdot p(n-1)$ (mit dem Anfangswert $p(1)=1$).\\
	Hinweis: Betrachten sie die Matrixdarstellung, so muss die ''1''   in der Position $(1,1)$ stehen
	und in der Position $(2,1)$ kann eine der $2n-1$ Zahlen $2 \leq k \leq 2n$ stehen. Das sollte
	Sie auf ein Verfahren füren, wie man alle Möglichkeiten konstruiert.
\item Zeigen Sie
	\[
	p(n) = \frac{(2n)!}{2^n \cdot n!}~~~~~(n \geq 1).
	\]
\item Verwenden Sie die Formel von \textsc{Stirling}, um eine Aussage über das
	asymptotische Verhalten der Folge $(p(n))_{n  \geq 1}$ zu erhalten. Folgt daraus,
	dass diese Folge \emph{nicht} C-rekursiv ist?
\item Angenommen, Sie sollen ein Sortierverfahren für perfekte Listen entwerfen, also für 
	Listen, bei denen die Grössenverhältnisse der Listenelemente denen einer perfekten
	Permutation entsprechen. Geben Sie eine \emph{untere Schranke} für die
	Anzahl der Vergleichsoperationen an (worst-case und average-case), die \emph{jeder}
	hierfür geeignete Algorithmus benötigt.
\end{flushenum}



%%% 
% Lösung
%%%

\subsection*{Lösung}

\begin{flushenum}
\newcommand{\lr}{\leftrightarrow}
% 1.
\item
	Die $15$ perfekten Permutationen für $\{1, 2, 3, 4, 5, 6\}$ sind:
	\[
		123456 \lr
		\begin{bmatrix}
			1 & 3 & 5 \\
			2 & 4 & 6\\
		\end{bmatrix} \quad
		123546 \lr
		\begin{bmatrix}
			1 & 3 & 4 \\
			2 & 5 & 6 \\
		\end{bmatrix} \quad
		123645 \lr
		\begin{bmatrix}
			1 & 3 & 4 \\
			2 & 6 & 5 \\
		\end{bmatrix}
	\]
	\[
		132456 \lr
		\begin{bmatrix}
			1 & 2 & 5 \\
			3 & 4 & 6 \\
		\end{bmatrix} \quad
		132546 \lr
		\begin{bmatrix}
			1 & 2 & 4 \\
			3 & 5 & 6 \\
		\end{bmatrix} \quad
		132645 \lr
		\begin{bmatrix}
			1 & 2 & 4 \\
			3 & 6 & 5 \\
		\end{bmatrix}
	\]
	\[
		142356 \lr
		\begin{bmatrix}
			1 & 2 & 5 \\
			4 & 3 & 6 \\
		\end{bmatrix} \quad
		142536 \lr
		\begin{bmatrix}
			1 & 2 & 3 \\
			4 & 5 & 6 \\
		\end{bmatrix} \quad
		142635 \lr
		\begin{bmatrix}
			1 & 2 & 3 \\
			4 & 6 & 5 \\
		\end{bmatrix}
	\]
	\[
		152346 \lr
		\begin{bmatrix}
			1 & 2 & 4 \\
			5 & 3 & 6 \\
		\end{bmatrix} \quad
		152436 \lr
		\begin{bmatrix}
			1 & 2 & 3 \\
			5 & 4 & 6 \\
		\end{bmatrix} \quad
		152634 \lr
		\begin{bmatrix}
			1 & 2 & 3 \\
			5 & 6 & 4 \\
		\end{bmatrix}
	\]
	\[
		162345 \lr
		\begin{bmatrix}
			1 & 2 & 4 \\
			6 & 3 & 5 \\
		\end{bmatrix} \quad
		162435 \lr
		\begin{bmatrix}
			1 & 2 & 3 \\
			6 & 4 & 5 \\
		\end{bmatrix} \quad
		162534 \lr
		\begin{bmatrix}
			1 & 2 & 3 \\
			6 & 5 & 4 \\
		\end{bmatrix}
	\]

%2.
\item
	Für die Position $\sigma_1$ gibt es nur eine Möglichkeit: $1$. Für $\sigma_2$ gibt es nur die Bedingung $\sigma_1 < \sigma_2$.
	Damit ist $\sigma_2$ eine beliebige Zahl der verbleibenden $\{2, \ldots 2n\}$, i.e. $2n-1$ Möglichkeiten.
	Für $\sigma_k$, $k \in \{3, \ldots 2n\}$ gelten die gleichen Bedingungen wie für $\sigma_j$, $j \in \{1, \ldots 2(n-1)\}$ und es gibt eine
	Bijektion zwischen $\{3, \ldots 2n\}$ und $\{1, \ldots 2(n-1)\}$. Damit folgt, dass es für die restlichen Positionen $\sigma_k$, $k \in \{3, \ldots 2n\}$
	$p(n-1)$ Möglichkeiten gibt. Insgesamt gibt es also $p(n) = (2n - 1)\cdot p(n-1)$ Möglichkeiten und eine perfekte Permutation kann rekursiv konstruiert werden.

%3.
\item
	Per Induktion zeigt man:
	\begin{enumerate}
		\item \textit{I.A.}: \[ p(1) = \frac{(2 \cdot 1)!}{2^1 \cdot 1!} = 1 \]
		\item \textit{I.V.}: \[ \exists n \in \mathbb{N}\backslash\{0\}: p(n) = \frac{(2n)!}{2^n \cdot n!} \]
		\item \textit{I.S.}: $ n \rightarrow n+1$: Mit Aufgabenteil 2. gilt:
			\[ p(n+1) = (2 (n+1) - 1) \cdot p(n) \overset{I.V.}{=} (2n + 1) \cdot \frac{(2n)!}{2^n \cdot n!} = \]
			\[ (2n + 1) \cdot \frac{(2n)!}{2^n \cdot n!} \cdot \frac{(2n + 1)(2n + 2)\cdot 2 \cdot (n+1)}{(2n+1)(2n + 2) \cdot2 \cdot (n+1)} = 
			   \frac{(2n+1) \cdot 2 \cdot (n+1)}{(2n + 1) (2n +2)}\frac{(2n+2)!}{2^{n+1} \cdot (n+1)!} \]
			\[ = \frac{(2(n+1))!}{2^{n+1}\cdot (n+1)!} \]
	\end{enumerate}

%4.
\item
	Mit der Formel von \textsc{Stirling} folgt:
	\[ p(n) \sim \frac{\left(\frac{2n}{e}\right)^{2n} \cdot \sqrt{2 \cdot \pi \cdot 2n}}{2^n \cdot \left(\frac{n}{e}\right)^n \cdot \sqrt{2 \cdot \pi \cdot n}} =
	   \frac{4^n \cdot \left(\frac{n}{e}\right)^n \cdot \sqrt{2}}{2^n} = 2^n \cdot \left(\frac{n}{e}\right)^n \cdot \sqrt{2} \]
	Dies ist keine C-Rekursion, da nicht $p(n) \in \mathcal{O}(n^k \lambda^n)$ für festes $k, \lambda$ gilt.
	\newcommand{\limninf}{\underset{n\rightarrow \infty}{\lim}}
	\[  \limninf \frac{n^k \cdot \lambda^n}{n^n} = \limninf \frac{\lambda^n}{n^{n-k}} = \limninf e^{n \ln \lambda - (n-k) \ln n} =
	    \limninf e^{-n(\ln n - \ln \lambda) + k \ln n} = 0 \]
	\[ \Rightarrow n^k \lambda^n \in \hbox{o}(n^n) \]

%5.
\item
	Da das Sortierverfahren vergleichsbasierend arbeiten muss entsteht ein Entscheidungsbaum mit $e = p(n)$ Blättern.
	Die Höhe dieses Baumes und damit die Laufzeit lässt sich nach unten abschätzen:
	\[h \geq \log e = \log p(n) = \log \left( 2^n \left(\frac{n}{e}\right)^n \sqrt{2}\right) = 
	 n \log \left( \frac{2n}{e}\right) + \frac{1}{2} \log 2 \in \Theta(n \log n)\]
	Die untere Schranke gilt offensichtlich sowohl für den worst-case als auch average-case.
\end{flushenum}
