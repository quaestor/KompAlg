\section*{Aufgabe 22 - Quickselect}
\addcontentsline{toc}{subsection}{Aufgabe 22 - Quickselect}
  \newcommand{\meanv}{\overline{v}}
  \lstset{ %
    language=Mathematica,
    basicstyle=\small,
    numbers=left,
    numberstyle=\small,
    numbersep=-9pt
  }
  \begin{lstlisting}
    QUICKSELECT[A_, k_] :=
      Module[
        {
          left, (* linke Teilliste *)
          right, (* rechte Teilliste *)
          p = RandomInteger[{1, Length[A]}] (* zuf. Pivot-Element *)
        },
        left = Select[A, # > A[[p]] &]; (* alle Elem. > Pivot *)
        right = Select[A, # < A[[p]] &]; (* alle Elem. < Pivot *)
        Return[
          With[
            (* Groesse von 'links' + { a | a = A[p] } *)
            {n = Length[A] - Length[right]},
            Which[
              (* gesuchtes Element ist in der linken Liste *)
              k <= Length[left], QUICKSELECT[left, k],
              (* gesuchtes Element ist in der rechten Liste *)
              k > n, QUICKSELECT[right, k - n],
              (* Pivot-Element ist bereits das Gesuchte *)
              True, A[[p]]
            ]
          ]
        ]
      ]
  \end{lstlisting}
\begin{enumerate}
\item
  Wird bei der Pivot-Wahl zufällig immer das größte Element gewählt, so wird
  die Liste in jedem Schritt lediglich um ein Element kleiner. Für die
  Aufteilung in die Teil-Listen sind bei Listenlänge $n$ genau $n-1$ Vergleiche
  notwendig. Insgesamt sind dies also $\sum_{i<n}i = \frac{1}{2}n(n-1) \in
  \mathcal{O}(n^2)$ Vergleiche.
\item
\[ \meanv(n) \leq \overbrace{(n-1)}^{(1)} +
\underbrace{\frac{1}{n}\sum_{1\leq k \leq n}}_{(3)}
\overbrace{\max\left\{\meanv(k-1),\meanv(n-k)\right\}}^{(2)} \]
\begin{itemize}
	\item[(1)] Anzahl der Vergleiche zur Aufteilung in die Teil-Listen
	\item[(2)] Da mit $\meanv(n) =
		\max_{1\leq k\leq n}\meanv(n,k)$ dasjenige $k$ gesucht ist, für das
		der mittlere Aufwand maximal ist, muss in dieser Summe
		ebenfalls aus den beiden Möglichkeiten (linke oder rechte
		Teil-Liste durchsuchen) das Maximum gewählt werden
	\item[(3)] Schließlich wird noch über alle möglichen Pivot-Elemente gemittelt
\end{itemize}
Zum Beweis des asymptotischen Verhaltens wird diese Ungleichung ein wenig
umgeformt. Sei o.B.d.A $n$ gerade. Da der Aufwand mit der Inputlänge
wächst ist
\[ \max\left\{\meanv(k-1), \meanv(n-k)\right\} =
\meanv(\max\{k-1,n-k\}). \]
Damit lässt sich die Summe auch schreiben als
\begin{align} \nonumber \sum_{1\leq k\leq n} \max\left\{\meanv(k-1),\meanv(n-k)\right\}
&= 2 \cdot \sum_{k=\frac{n}{2}+1}^n \meanv(k-1) = 2 \cdot \sum_{k=\frac{n}{2}}^{n-1} \meanv(k) \\
\intertext{oder, um angenehmere Grenzen zu erhalten}
&= 2 \cdot \left(\sum_{k=0}^{n-1} \meanv(k) - \sum_{k=0}^{\frac{n}{2}-1} \meanv(k)\right) \label{eqn:sum}
\end{align}
Nun können wir per Induktion zeigen, dass $\meanv(n) \in \mathcal{O}(n)$:
\begin{itemize}
	\item \textit{Induktionsanfang:} \[ \meanv(0) = 0 \leq c \cdot 0 \]
		wird von beliebigen positiven Konstanten $c$ erfüllt.
	\item \textit{Induktionsannahme:} \[ \meanv(k) \leq c \cdot k \text{ für } k < n \]
		für ein festes, positives $c$
	\item \textit{Induktionsschritt:} mit Gleichung \eqref{eqn:sum} folgt
		\begin{align*}
			\meanv(n) &\leq (n-1) + \frac{2}{n}
			\left(\sum_{k=0}^{n-1} \meanv(k) -
			\sum_{k=0}^{\frac{n}{2}-1} \meanv(k)\right) \\
			\intertext{Dies wird unter Verwendung der Induktionsannahme zu}
			&\leq (n-1) + \frac{2}{n} \left(\sum_{k=0}^{n-1}
			c \cdot k - \sum_{k=0}^{\frac{n}{2}-1} c \cdot k\right) \\
			&= (n-1) + \frac{2\cdot c}{n} \left(\frac{n(n-1)}{2}-\frac{n(\frac{n}{2}-1)}{4}\right) = (n-1) + \frac{c}{4} (3n - 2)
		\end{align*}
		Wählt man nun z.B. $c = 4$, so ist die Ungleichung
			\[ (n-1) + \frac{c}{4} (3n - 2) \leq c \cdot n \]
			\[ 4n-3 \leq 4n \]
		offensichtlich erfüllt und es folgt die Behauptung.
\end{itemize}
\end{enumerate}
