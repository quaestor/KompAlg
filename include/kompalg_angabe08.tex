%%%
% Angabe: Die Begleitmatrix einer linearen Rekursion
%%%

Ist ${\mathbf a} = (a_1,a_2,\ldots,a_k)$ der Vektor der Rekursionskoeffizienten zu
der linearen Rekursion der Ordnung $k$
\begin{equation*}
x_{n} = a_1 \, x_{n-1} + a_2 \, x_{n-2} + \cdots + a_k \, x_{n-l}~~~(n \geq k)
\tag{$*$}
\end{equation*}
so bezeichnet man die $(k \times k)$-Matrix
\[
C_{{\mathbf a}} = \begin{bmatrix}
0 & 0 & 0 &    \ldots & 0& 0& 0  & a_k \cr
1 & 0 & 0 &    \ldots & 0& 0& 0  & a_{k-1} \cr
0 & 1 & 0 &   \ldots & 0& 0& 0 & a_{k-2} \cr
0 & 0 & 1 &   \ldots & 0& 0& 0 & a_{k-3} \cr
\vdots & \vdots    & \vdots & \ddots & \vdots&  \vdots &\vdots & \vdots \cr
0 & 0 & 0 &  \ldots &1 & 0 & 0 & a_3 \cr
0 & 0 & 0 &    \ldots &0 & 1& 0 & a_2 \cr
0 & 0 & 0 &   \ldots &0 & 0& 1 & a_1
\end{bmatrix}
\]
als die \emph{Begleitmatrix} (engl. \emph{companion matrix}) zu der Rekursion $(*)$.

\begin{flushenum}
\item Zeigen Sie: ist ${\mathbf x} = (x_n)_{n \geq 0}$  eine Folge, die der Rekursion $(*)$
	genügt, also  ${\mathbf x} \in \mathcal{V}_{{\mathbf a}}$, 
	und bezeichnet man für jedes $n \geq 0$ mit ${\mathbf x}^{(n)}$ den Abschnitt
	\[
	{\mathbf x}^{(n)} = \begin{bmatrix} x_n & x_{n+1}  & x_{n+2} & \ldots & x_{n+k-1} \end{bmatrix}
	\]
	der Folge ${\mathbf x}$, so gilt für jedes $n \geq 0$:
	\[
	{\mathbf x}^{(n+1)} = {\mathbf x}^{(n)} \cdot C_{{\mathbf a}}
	\]
	und somit auch (per Induktion)
	\[
	{\mathbf x}^{(n)} = {\mathbf x}^{(0)} \cdot C_{{\mathbf a}}^n = 
	\begin{bmatrix} x_{0} & x_{1}  & x_{2} & \ldots & x_{k-1} \end{bmatrix}\cdot C_{{\mathbf a}}^n .
	\]
\item Zeigen Sie, dass das charakteristische Polynom der Matrix $C_{{\mathbf a}}$ 
	gleich dem charakteristischen Polynom der Rekursion $(*)$ ist, d.h.
	\[
	\chi_{C_{{\mathbf a}}}(z) = \chi_{{\mathbf a}}(z),
	\]
	also explizit geschrieben
	\[
	\det (z \cdot I - C_{{\mathbf a}}) = z^k - a_1 z^{k-1}- \cdots - a_k.
	\]
	\underline{Hinweis}: $\det (z \cdot I - C_{{\mathbf a}})$ nach der letzten Spalte entwickeln.
	
\item Für  komplexe Zahlen $\alpha_1, \ldots , \alpha_k$ wird mit
	$V(\alpha_1,\alpha_2,\ldots,\alpha_k)$ die \textsc{Vandermonde}-Matrix
	\[
	V(\alpha_1,\alpha_2,\ldots,\alpha_k) = 
		\begin{bmatrix}
			\alpha_i^{j-1}
		\end{bmatrix}_{1 \leq i,j \leq k}
	= \begin{bmatrix}
	1 & \alpha_1 & \alpha_1^2 & \ldots & \alpha_1^{k-1} \cr
	1 & \alpha_2 & \alpha_2^2 & \ldots & \alpha_2^{k-1} \cr
	\vdots & \vdots & \vdots & \ddots & \vdots \cr
	1 & \alpha_k & \alpha_k^2 & \ldots & \alpha_k^{k-1} 
	\end{bmatrix}
	\]
	bezeichnet.
	
	Jetzt wird angenommen, dass das charakteristische Polynom $\chi_{{\mathbf a}}(z)$
	$k$ verschiedene Nullstellen $\lambda_1,\lambda_2,\ldots,\lambda_k$ hat, das sind 
	wegen 2. die Eigenwerte von der Begleitmatrix $C_{{\mathbf a}}$.
	
	Zeigen Sie (Teil 1. beachten), dass
	\[
	V(\lambda_1,\lambda_2,\ldots,\lambda_k) \cdot C_{{\mathbf a}} =
	diag(\lambda_1,\lambda_2,\ldots,\lambda_k) \cdot V(\lambda_1,\lambda_2,\ldots,\lambda_k)
	\]
	gilt, wobei die Diagonalmatrix $diag(\lambda_1,\lambda_2,\ldots,\lambda_k)$ der Eigenwerte ist.
	Anders geschrieben:
	\[
	V(\lambda_1,\lambda_2,\ldots,\lambda_k) \cdot C_{{\mathbf a}} \cdot V(\lambda_1,\lambda_2,\ldots,\lambda_k)^{-1}= 
	diag(\lambda_1,\lambda_2,\ldots,\lambda_k)
	\]
	d.h. die \textsc{Vandermonde}-Matrix $V(\lambda_1,\lambda_2,\ldots,\lambda_k)$
	diagonalisiert die Begleitmatrix $C_{{\mathbf a}}$.
	
	
	\underline{Hinweis}:  Die $k$ geometrischen Folgen 
	${\boldsymbol \lambda}_j = (1,\lambda_j,\lambda_j^2, \ldots)$
	$(1 \leq j \leq k)$ genügen der Rekursion $(*)$.

\end{flushenum}
