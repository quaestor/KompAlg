\section*{Eine weitere schnelle Transformation}
\addcontentsline{toc}{subsection}{Eine weitere schnelle Transformation}
%%%
% Angabe
%%%
\subsection*{Angabe}
Für $k \geq 1$ bezeichne $\mathcal{H}_k$ die reelle $(2^k \times 2^k)$-Matrix,
deren Zeilen und Spalten mit den $2^k$ binären Vektoren aus
$\mathbb{B}^k=\{0,1\}^k$ indiziert sind und deren Einträge sich mittels des
Skalarproduktes für Bitvektoren
\[
\mathbf{u}\cdot\mathbf{v}= \sum_{1 \leq i \leq n} u_i v_i
~~~(\mathbf{u}=u_1u_2\ldots u_k, \mathbf{v}=v_1v_2\ldots v_k \in \mathbb{B}^k)
\]
wie folgt darstellen:
\[
\left( \mathcal{H}_k\right)_{\mathbf{u},\mathbf{v}} =
(-1)^{\mathbf{u}\cdot \mathbf{v}}~~~\text{für}~
\mathbf{u},\mathbf{v} \in \mathbb {B}^k
\]
Beachten Sie: es ist egal, ob man das Skalarprodukt binär oder reell
ausrechnet. Auf alle Fälle sollen die Einträge der Matrix, die $\pm 1$ sind,
als reelle Zahlen interpretiert werden.
Für $k=0$ setzt man $\mathcal{H}_0 = (1)$. Es ist sinnvoll anzunehmen, dass die
Elemente von $\mathbb{B}^k$ in lexikografischer Ordnung zum Indizieren
herangezogen werden -- damit die nachfolgenden Formeln stimmen.

Beispiele (mit Zeilen- und Spaltenindizierung)
\begin{align*}
n=1&&
\begin{array}{c|rr}
   & 0 & 1 \\ \hline
0 & 1 & 1 \cr
1 & 1 & -1
\end{array}
&\text{~~also~~}
\mathcal{H}_1 = \begin{bmatrix} 1 & 1 \cr 1 & -1 \end{bmatrix}
\cr
n=2&&
\begin{array}{c|rrrr}
      & 00 & 01 & 10 & 11 \\ \hline
 00 & 1 & 1 & 1 & 1 \cr
 01 & 1 & -1 & 1 & -1 \cr
 10 & 1 & 1 & -1 & -1 \cr
 11 &  1 & -1 & -1 & 1 
 \end{array}
 &\text{~~also~~}
\mathcal{H}_2 =
\begin{bmatrix}
	1 & 1 & 1 & 1 \cr
	1 & -1 & 1 & -1\cr
	1 & 1 & -1 & -1 \cr
	1 & -1 & -1 & 1
\end{bmatrix}
\cr
n=3&&
\begin{array}{c|rrrrrrrr}
     & 000 & 001 & 010 & 011& 100 & 101 & 110 & 111 \\ \hline
 000 & 1 & 1  & 1  & 1  & 1 & 1  & 1  & 1 \cr
 001 & 1 & -1 & 1  & -1 &1  & -1 & 1  & -1 \cr
 010 & 1 & 1  & -1 & -1 & 1 & 1  & -1  & -1 \cr
 011 &  1 & -1 & -1 & 1 &  1 & -1 & -1 & 1 \cr
 100 & 1 & 1 & 1 & 1 & -1 & -1 & -1 & -1 \cr
 101 & 1 & -1 & 1 & -1 & -1 & 1 & -1 & 1 \cr
 110 & 1 & 1 & -1 & -1  & -1 & -1 & 1 & 1 \cr
 111 &  1 & -1 & -1 & 1 & -1 & 1 & 1 & -1
 \end{array}
 &
 \end{align*}

\begin{flushenum}
\item Zeigen Sie, dass für $k \geq 0$ stets gilt:
\[
\mathcal{H}_{k+1} =
\begin{bmatrix}
\mathcal{H}_k & \mathcal{H}_k \cr
\mathcal{H}_k & - \mathcal{H}_k
\end{bmatrix}
\]

\item
Entwerfen Sie einen Algorithmus, der diese rekursive Struktur ausnutzt und es
erlaubt, die lineare Transformation von (Spalten-)Vektoren $\mathbf{a} \in
\mathbb{R}^{2^k}$
\[
\mathcal{H}_k :
\mathbb{R}^{2^k}\longrightarrow \mathbb{R}^{2^k}: 
\mathbf{a} \longmapsto\mathcal{H}_k \cdot \mathbf{a}
\]
mit $k \cdot 2^k$ reellen Additionen und Subtraktionen zu berechnen --- und
damit wesentlich effizienter, als wenn man das Produkt $\mathcal{H}_k \cdot
\mathbf{a}$ nach üblicher (Schul-)Methode ausrechnet.

\item Zeigen Sie (am einfachsten per Induktion), dass die Zeilen der Matrix
	$\mathcal{H}_k$ paarweise orthogonal sind, genauer:
\[
\mathcal{H}_k^2 = \mathcal{H}_k  \cdot \mathcal{H}_k^t = 2^k \,\, \mathbb{I}_k,
\]
wobei $\mathbb{I}_k$ die Einheitsmatrix vom Format ist.  Anders gesagt: die
Matrix $2^{k/2}\, \mathcal{H}_k$ ist eine \emph{orthogonale} Matrix und die
Transformation in 2. eine \emph{orthogonale} Transformation.
\end{flushenum}

Eine weiterführende Bemerkung zu dieser Aufgabe: Sollte Sie diese Situation
irgendwie an die DFT und FFT erinnern, so hat das sehr viel für sich.  Man
nennt die Matrizen $\mathcal{H}_k$ (spezielle) \textsc{Hadamard}-Matrizen und
die Transformation von Vektoren (wie in Teil 2.) die
\textsc{Fourier-Hadamard}-Transformation.  Die Rolle der zyklischen Gruppen bei
der Fourier-Transformation wird hier von (den nicht-zyklischen) Gruppen
$(\mathbb{F}_2^k,\oplus)$ übernommen -- aber sonst geht man ganz analog vor.
Diese Matrizen spielen beim Quantencomputing eine ganz wichtige Rolle!


%%%
% Lösung
%%%
\subsection*{Lösung}
\renewcommand{\H}{\mathcal{H}}
\begin{flushenum}
% aufgabe 1.
	\item Z.z.: Für alle $k \geq 0$ gilt:
	\[ \H_{k+1} = \begin{bmatrix}
		\H_k & \H_k \\
		\H_k & -\H_k \\
	\end{bmatrix} \]
	Mit der Definition, dass die Einträge der Hadamardmatrix $\H_k$ über die Indizierung
	mittels der Binärvektoren der Länge $k$ bestimmt sind
	\[ \left( \H_k \right)_{u,v} = (-1)^{u \cdot v} \text{ für } u,v \in \mathds{B}^k \]
	lässt sich die Matrix $\H_{k+1}$ schreiben als:
	\[ \H_{k+1} = \begin{bmatrix}
		A & B \\
		C & D \\
	\end{bmatrix} \]
	Mit
	\[ \left( A \right)_{0u,0v} = \left( B\right)_{1u, 0v} = \left( C \right)_{0u, 1v} = (-1)^{u \cdot v} = \H_k \]
	\[ \left( D \right)_{1u, 1v} = (-1)^{1 + u \cdot v} = - \H_k \]
	stimmt also die Behauptung.

% aufgabe 2.
	\item Nach Aufgabenteil (1) gilt für jedes $k \geq 0$
	\[ \H_{k+1} = \begin{bmatrix}
		\H_k & \H_k \\
		\H_k & -\H_k \\
	\end{bmatrix} \]
	Damit kann man die Multiplikation von $\H_k \cdot a$ rekursiv so definieren:
	\[ \H_k \cdot a = 
	\begin{bmatrix}
		\H_{k-1} & \H_{k-1} \\
		\H_{k-1} & -\H_{k-1} \\
	\end{bmatrix} \cdot
	\begin{bmatrix}
		a_{upper} \\
		a_{lower}
	\end{bmatrix} = 
	\begin{bmatrix}
		\H_{k-1} \cdot (a_{upper} + a_{lower}) \\
		\H_{k-1} \cdot (a_{upper} - a_{lower}) \\
	\end{bmatrix} \]
	Durch Aufspaltung des Vektors $a$ in obere und untere Hälfte erhält man
	eine Rekursion für den Aufwand in Abhängigkeit des Parameters $k$. Es
	müssen in jedem Rekursions\-schritt zwei Matrixmultiplikationen der
	halben Größe (i.e. $T(k-1)$) sowie $2^k$ Additionen bzw. Subtraktionen
	durchgeführt werden:
	\[ T(k) = 2 T(k-1) + 2^k \]

	\pagebreak

	Induktion:
	\begin{itemize}
		\item \textit{I.A.}: $k = 0$: $\H_k = [1] \Rightarrow \H_0 \cdot a = a$ benötigt $0 = 0\cdot 2^0 = k\cdot 2^k$ Operationen.

		\item \textit{I.V.}: Für ein $k$ gelte $T(k) = k\cdot 2^k$

		\item \textit{I.S.}: $k \rightarrow k + 1$:
			\[T(k+1) = 2 T(k) + 2^{k+1} \overset{I.V.}{=} 2\cdot k\cdot 2^k + 2^{k+1} = 2 (k+1)2^k = (k+1) 2^{k+1} \]
			Somit erfüllt der Algorithmus die gewünschte Bedingung.
	\end{itemize}

% aufgabe 3.
	\item Induktion:
		\begin{itemize}
			\item \textit{I.A.}: Für $k = 0$ gilt 
				\[ \H_0^2 = \H_0 \cdot \H_0^T = [1] = 2^0 \mathds{1}_0\]
	
			\item \textit{I.V.}: Für ein $k$ gelte:
				\[ \H_k^2 = \H_k \cdot \H_k^T = 2^k \mathds{1}_k \]

			\item \textit{I.S.}: $k \rightarrow k + 1$, mit Aufgabenteil (1):
				\begin{eqnarray*}
					\H_{k+1}^2 &=& \H_{k+1} \cdot \H_{k+1}^T = \begin{bmatrix}
						\H_k & \H_k \\
						\H_k & -\H_k \\
					\end{bmatrix} \cdot \begin{bmatrix}
						\H_k & \H_k \\
						\H_k & -\H_k \\
					\end{bmatrix}^T =
					\begin{bmatrix}
						2 \H_k^2 & 0 \\
						0 & 2 \H_k^2
					\end{bmatrix} \overset{I.V.}{=} \\
					&=&
					\begin{bmatrix}
						2^{k+1} \mathds{1}_k & 0 \\
						0 & 2^{k+1}\mathds{1}_k \\
					\end{bmatrix} = 
					2^{k+1} \mathds{1}_{k+1}
				\end{eqnarray*}
		\end{itemize}
\end{flushenum}
