\section*{Aufgabe 12 - Eine weitere schnelle Transformation}
\addcontentsline{toc}{subsection}{Aufgabe 12 - Eine weitere schnelle Transformation}

\renewcommand{\H}{\mathcal{H}}
\begin{enumerate}[1.]
	\item Z.z.: Für alle $k \geq 0$ gilt:
	\[ \H_{k+1} = \begin{bmatrix}
		\H_k & \H_k \\
		\H_k & -\H_k \\
	\end{bmatrix} \]
	Mit der Definition, dass die Werte der Hadamardmatrix $\H_k$ über die Indizierung 
	mittels der Binärvektoren der Länge $k$ bestimmt sind
	\[ \left( \H_k \right)_{u,v} = (-1)^{u \cdot v} \text{ für } u,v \in \mathds{B}^k \]
	lässt sich die Matrix $\H_{k+1}$ schreiben als:
	\[ \H_{k+1} = \begin{bmatrix}
		A & B \\
		C & D \\
	\end{bmatrix} \]
	Mit
	\[ \left( A \right)_{0u,0v} = \left( B\right)_{1u, 0v} = \left( C \right)_{0u, 1v} = (-1)^{u \cdot v} = \H_k \]
	\[ \left( D \right)_{1u, 1v} = (-1)^{1 + u \cdot v} = - \H_k \]
	stimmt also die Behauptung.

	\item Nach Aufgabenteil (1) gilt für jedes $k \geq 0$
	\[ \H_{k+1} = \begin{bmatrix}
		\H_k & \H_k \\
		\H_k & -\H_k \\
	\end{bmatrix} \]
	Damit kann man die Multiplikation von $\H_k \cdot a$ rekursiv so definieren:
	\[ \H_k \cdot a = 
	\begin{bmatrix}
		\H_{k-1} & \H_{k-1} \\
		\H_{k-1} & -\H_{k-1} \\
	\end{bmatrix} \cdot
	\begin{bmatrix}
		a_{upper} \\
		a_{lower}
	\end{bmatrix} = 
	\begin{bmatrix}
		\H_{k-1} \cdot (a_{upper} + a_{lower}) \\
		\H_{k-1} \cdot (a_{upper} - a_{lower}) \\
	\end{bmatrix} \]
	Durch Aufspaltung des Vektors $a$ in obere und untere Hälfte erhält man
	eine Rekursion für den Aufwand in Abhängigkeit des Parameters $k$. Es
	müssen in jedem Rekursionsschritt zwei Matrixmultiplikationen der
	halben Größe (i.e. $T(k-1)$) sowie $2^k$ Additionen/Subtraktionen
	durchgeführt werden:
	\[ T(k) = 2 T(k-1) + 2^k \]
	Induktion:

	\begin{itemize}
		\item \textit{I.A.}: $k = 0$: $\H_k = [1] \Rightarrow \H_0 \cdot a = a$ benötigt $0 = 0 2^0 = k 2^k$ Operationen.

		\item \textit{I.V.}: Für ein $k$ gelte $T(k) = k 2^k$

		\item \textit{I.S.}: $k \rightarrow k + 1$:
			\[T(k+1) = 2 T(k) + 2^{k+1} \overset{I.V.}{=} 2 k 2^k + 2^{k+1} = 2 (k+1)2^k = (k+1) 2^{k+1} \]
			Somit erfüllt der Algorithmus die gewünschte Bedingung.
	\end{itemize}

	\item Induktion:
		\begin{itemize}
			\item \textit{I.A.}: Für $k = 0$ gilt 
				\[ \H_0^2 = \H_0 \cdot \H_0^T = [1] = 2^0 \mathds{1}_0\]
	
			\item \textit{I.V.}: Für ein $k$ gelte:
				\[ \H_k^2 = \H_k \cdot \H_k^T = 2^k \mathds{1}_k \]

			\item \textit{I.S.}: $k \rightarrow k + 1$, mit Aufgabenteil (1):
				\begin{eqnarray*}
					\H_{k+1}^2 &=& \H_{k+1} \cdot \H_{k+1}^T = \begin{bmatrix}
						\H_k & \H_k \\
						\H_k & -\H_k \\
					\end{bmatrix} \cdot \begin{bmatrix}
						\H_k & \H_k \\
						\H_k & -\H_k \\
					\end{bmatrix}^T =
					\begin{bmatrix}
						2 \H_k^2 & 0 \\
						0 & 2 \H_k^2
					\end{bmatrix} \overset{I.V.}{=} \\
					&=&
					\begin{bmatrix}
						2^{k+1} \mathds{1}_k & 0 \\
						0 & 2^{k+1}\mathds{1}_k \\
					\end{bmatrix} = 
					2^{k+1} \mathds{1}_{k+1}
				\end{eqnarray*}
		\end{itemize}

\end{enumerate}
