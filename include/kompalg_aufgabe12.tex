

\section*{Aufgabe 12 - Eine weitere schnell Transformation}
\addcontentsline{toc}{subsection}{Aufgabe 12 - Eine weitere schnell Transformation}

\begin{enumerate}[1.]
	\item Z.z.: Für alle $k \geq 0$ gilt:
	\[ \mathcal{H}_{k+1} = \begin{bmatrix}
		\mathcal{H}_k & \mathcal{H}_k \\
		\mathcal{H}_k & -\mathcal{H}_k \\
	\end{bmatrix} \]
	Mit der Definition, dass die Werte der Hadamardmatrix $\mathcal{H}_k$ über die Indizierung 
	mittels der Binärvektoren der Länge $k$ bestimmt sind
	\[ \left( \mathcal{H}_k \right)_{u,v} = (-1)^{u \cdot v} \text{ für } u,v \in \mathds{B}^k \]
	lässt sich die Matrix $\mathcal{H}_{k+1}$ schreiben als:
	\[ \mathcal{H}_{k+1} = \begin{bmatrix}
		A & B \\
		C & D \\
	\end{bmatrix} \]
	Mit
	\[ \left( A \right)_{0u,0v} = \left( B\right)_{1u, 0v} = \left( C \right)_{0u, 1v} = (-1)^{u \cdot v} = \mathcal{H}_k \]
	\[ \left( D \right)_{1u, 1v} = (-1)^{1 + u \cdot v} = - \mathcal{H}_k \]
	stimmt also die Behauptung.

	\item Nach Augabenteil (1) gilt für jedes $k \geq 0$
	\[ \mathcal{H}_{k+1} = \begin{bmatrix}
		\mathcal{H}_k & \mathcal{H}_k \\
		\mathcal{H}_k & -\mathcal{H}_k \\
	\end{bmatrix} \]
	Damit kann man die Multiplikation von $\mathcal{H}_k \cdot a$ rekursiv so definieren:
	\[ \mathcal{H}_k \cdot a = 
	\begin{bmatrix}
		\mathcal{H}_{k-1} & \mathcal{H}_{k-1} \\
		\mathcal{H}_{k-1} & -\mathcal{H}_{k-1} \\
	\end{bmatrix} \cdot
	\begin{bmatrix}
		a_{upper} \\
		a_{lower}
	\end{bmatrix} = 
	\begin{bmatrix}
		\mathcal{H}_{k-1} \cdot (a_{upper} + a_{lower}) \\
		\mathcal{H}_{k-1} \cdot (a_{upper} - a_{lower}) \\
	\end{bmatrix} \]
	Durch Aufspaltung des Vectors $a$ in obere und untere Hälfte erhält man so eine Rekursion für den Aufwand
	in Abhängigkeit des Parameters $k$, es müssen in jedem Rekursionsschritt zwei Matrixmultiplikationen der halben
	Größe (i.e. $k-1$) sowie $2^k$ Additionen/Subtraktionen durchgeführt werden:
	\[ T(k) = 2 T(k-1) + 2^k \]
	Induktion:

	I.A.: $k = 0$: $\mathcal{H}_k = [1] \Rightarrow \mathcal{H}_0 \cdot a = a$ benötigt $0 = 0 2^0 = k 2^k$ Operationen.

	I.V.: Für ein $k$ gelte $T(k) = k 2^k$

	I.S.: $k \rightarrow k + 1$:
	\[T(k+1) = 2 T(k) + 2^{k+1} \overset{I.V.}{=} 2 k 2^k + 2^{k+1} = 2 (k+1)2^k = (k+1) 2^{k+1} \]
	Somit erfüllt der Algorithmus die gewünschte Bedingung.

	\item Induktion:
	
	I.A.: Für $k = 0$ gilt 
	\[ \mathcal{H}_0^2 = \mathcal{H}_0 \cdot \mathcal{H}_0^T = [1] = 2^0 \mathds{1}_0\]
	
	I.V.: Für ein $k$ gelte:
	\[ \mathcal{H}_k^2 = \mathcal{H}_k \cdot \mathcal{H}_k^T = 2^k \mathds{1}_k \]

	I.S.: $k \rightarrow k + 1$, mit Aufgabenteil (1):
	\begin{eqnarray*}
	\mathcal{H}_{k+1}^2 &=& \mathcal{H}_{k+1} \cdot \mathcal{H}_{k+1}^T = \begin{bmatrix}
		\mathcal{H}_k & \mathcal{H}_k \\
		\mathcal{H}_k & -\mathcal{H}_k \\
	\end{bmatrix} \cdot \begin{bmatrix}
		\mathcal{H}_k & \mathcal{H}_k \\
		\mathcal{H}_k & -\mathcal{H}_k \\
	\end{bmatrix}^T =
	\begin{bmatrix}
		2 \mathcal{H}_k^2 & 0 \\
		0 & 2 \mathcal{H}_k^2
	\end{bmatrix} \overset{I.V.}{=} \\
	&=&
	\begin{bmatrix}
		2^{k+1} \mathds{1}_k & 0 \\
		0 & 2^{k+1}\mathds{1}_k \\
	\end{bmatrix} = 
	2^{k+1} \mathds{1}_{k+1}
	\end{eqnarray*}

\end{enumerate}
