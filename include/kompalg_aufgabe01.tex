\section*{Fingerübungen zum Rechnen mit Logarithmen}
\addcontentsline{toc}{subsection}{Fingerübungen zum Rechnen mit Logarithmen}
\subsection*{Angabe}
%%%
% Angabe: Fingerübungen zum Rechnen mit Logarithmen
%%%

Bemerkung: Der sichere Umgang mit Logarithmen (und anderen elementaren
Funktionen der Mathematik, wie Polynome, Exponentialfunktion,
Winkelfunktionen) ist nicht Stoff der Veranstaltung,
sondern gehiört zum Handwerkszeug, das man vorab beherrschen sollte.
Die nachfolgenden Aufgaben sind als Test zu verstehen: wer mit so etwas
ernsthaft Schwierigkeiten hat, hat etwas nachzuholen.
Diese Aufgaben werden nicht unbedingt alle in der Übungsgruppe behandelt.

Bezeichnung: für eine relle Zahl $b>1$ und positives reelles $y$ bezeichnet
$\log_b y$  den Logarithmus von $y$ zur Basis $b$: das ist die eindeutig
bestimmte reelle Zahl $x$ mit $b^x = y$.  Logarithmen zur Basis $b=2$ werden
oft einfach mit $\log y$ oder $\lg y$ bezeichnet, ``natürliche'' Logarithmen
zur Basis $e$ mit $\ln y$.

\begin{flushenum}
\item
Ist $c$ eine weitere positive reelle Zahl, wie  drückt sich $\log_c y$
mittels Logarithmen zur Basis $b$ aus?
\item
In welchem Verhältnis stehen $\log_{10}2$ und $\log_2 10$ zueinander?
\item
Wieviele Binärstellen genügen in jedem Fall, 
um eine 14-stellige Dezimalzahl im Binärsystem darzustellen?
\item
Wieviele Dezimalstellen hat die Zahl\footnote{
N.B. Mit der Faktorisierung dieser Zahl in ihre zwei Primfaktoren
durch M. Morrison und J. Brillhard 
im Jahr 1970 begann die Neuzeit des Faktorisierens.} $2^{128}+1$ ?
\item
\"Uberzeugen Sie sich erst einmal durch Beispiele 
(mittels Taschenrechner, Maple, Logarithmentafel, \ldots) davon, 
dass für beliebiges $x >0$ die Zahlen
$\log_2 x$ und $\ln x + \log_{10} x$ verblüffend nahe beieinander liegen. 
Wie gross ist der relative Fehler?
\item
Wie drückt sich $\log_b \log_b x$ mittels natürlicher Logarithmen aus?
\item
An welcher Stelle nimmt die Funktion 
$x \mapsto x \cdot \log_b x$ für $x \in \mathbb{R}_+$
ihr Minimum an? 
\item
$x$ sei eine reelle Zahl $>1$. An welcher Stelle nimmt die Funktion 
$b \mapsto b \cdot \log_b x$ für $b \in \mathbb{R}_+$
ihr Minimum an? 
\item
Was ist das Minimum der Funktion 
$x \mapsto x \cdot \sqrt[x]{n}$ für $x \in \mathbb{R}_+, n \in \mathbb{N}$?
\item
Zeigen Sie, dass $\log_{10} 2$ keine rationale Zahl ist.

\end{flushenum}




%%%
% Lösung
%%%
\subsection*{Lösung}
\begin{flushenum}
%1
\item
	\[ c^{\log_c y} = y \Leftrightarrow 
	\log_b c^{\log_c y} = \log_b y \Leftrightarrow \log_c y =
	\frac{\log_b y}{\log_b c}\]
%2
\item Nach Aufgabenteil 1.:
	\[ \log_{10} 2 = \frac{\log_2 2}{\log_2 10} = \frac{1}{\log_2 10}
	\Rightarrow \frac{\log_{10} 2}{\log_2 10} = \frac{1}{\log_2^2 10} \]
%3
\item 47 Ziffern \[ \log_2 10^{14} = 14 \log_2 10 \approx 46.5 \]
%4
\item 39 Ziffern \[ \log_{10} (2^{128} + 1) \approx 128 \log_{10} 2 \approx 38.5 \]
%5
\item Der relative Fehler ist unabhängig von $x$:
	\[ \left| \frac{\log_2 x - \ln x - \log_{10} x}{\log_2 x} \right| =
	\left| \frac{\frac{\ln x}{\ln 2} - \ln x - \frac{\ln x}{\ln
	10}}{\frac{\ln x}{\ln 2}} \right| = \frac{\ln 2}{\ln 10} \]
%6
\item
	\[ \log_b \log_b x = \frac{\ln \log_b x}{\ln b} = \frac{\ln \frac{\ln
	x}{\ln b}}{\ln b} = \frac{\ln \ln x - \ln \ln b}{\ln b}\]
%7
\item Da $b > 1$ ist das Minimum der Funktion $f(x) = x \cdot \log_b x$ bei $x = e^{-1}$:
	\[ f'(x) = \frac{1}{\ln b} + \frac{\ln x}{\ln b} \]
	\[ f''(x) = \frac{1}{x \cdot \ln b} \]
	Die zweite Ableitung ist bei $x = e^{-1}$ positiv, somit ist dort das Minimum.
%8
\item Für $x > 1$ nimmt die Funktion $f(b) = b \cdot \log_b x$ ihr Minimum bei $b = e$ an:
	\[ f'(b) = -\frac{\ln x}{\ln^2 b} + \frac{\ln x}{\ln b} \]
	\[f''(b) = \frac{2 \cdot \ln x}{b \cdot \ln^3 b} - \frac{\ln x}{b \cdot \ln^2 b} \]
	Die zweite Ableitung ist positiv, also Minimum.
%9
\item Das Minimum der Funktion $f(x) = x \cdot n^{1/x}$ wird bei $x = \ln n$ erreicht:
	Ist $n = 0$ gibt es kein Minimum in engeren Sinn, da die Funktion $f \equiv 0$ ist.
	Da $x > 0$ gilt ist fü $n>0$ die Funktion immer positiv und man kann den Logarithmus dieser Funktion betrachen um die Nullstelle zu finden;
	da der Logarithmus streng monoton steigend ist, ist die Nullstelle von $\ln f(x)$ equivalent mit der Nullstelle von $f(x)$
	\[ g(x) = \ln f(x) = \ln x + \frac{1}{x} \ln n \]
	\[ g'(x) = \frac{1}{x} - \frac{\ln n}{x^2} \]
	\[ g''(x) = -\frac{1}{x^2} + \frac{2 \cdot \ln n}{x^3} \]
	Die zweite Ableitung ist bei $x = \ln n$ positiv, also ist hier das Minimum.
%10
\item Annahme $\log_{10} 2 = \frac{p}{q} \in \mathbb{Q}$ mit $p,q \in
\mathbb{N}$ ($\log_{10} 2$ > 0, da die Logarithmusfunktion streng monoton
steigend ist und bei $1$ eine Nullstelle hat). Dann folgt:

\[ 10^{\frac{p}{q}} = 2 \Leftrightarrow 10^p = 5^p 2^p = 2^q \]

Die linke Seite hat Primteiler $5$, die rechte jedoch nicht; da die
Primteilerzerlegung eindeutig ist führt dies zu einem Widerspruch, also ist
$\log_{10} 2 \not\in \mathbb{Q}$.
\end{flushenum}

