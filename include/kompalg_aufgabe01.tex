\section*{Aufgabe 1 - Fingerübungen zum Rechnen mit Logarithmen}
\addcontentsline{toc}{subsection}{Aufgabe 1 - Fingerübungen zum Rechnen mit Logarithmen}

\begin{enumerate}
\item
	\[ c^{\log_c y} = y \Leftrightarrow 
	\log_b c^{\log_c y} = \log_b y \Leftrightarrow \log_c y =
	\frac{\log_b y}{\log_b c}\]
\item Nach Aufgabenteil 1.:
	\[ \log_{10} 2 = \frac{\log_2 2}{\log_2 10} = \frac{1}{\log_2 10} \Rightarrow \frac{\log_{10} 2}{\log_2 10} = \frac{1}{\log_2^2 10} \]

\item 47 Ziffern \[ \log_2 10^{14} = 14 \log_2 10 \approx 46.5 \]

\item 39 Ziffern \[ \log_{10} (2^{128} + 1) \approx 128 \log_{10} 2 \approx 38.5 \]

\item Der relative Fehler ist unabhängig von $x$:
	\[ \left| \frac{\log_2 x - \ln x - \log_{10} x}{\log_2 x} \right| =
	\left| \frac{\frac{\ln x}{\ln 2} - \ln x - \frac{\ln x}{\ln 10}}{\frac{\ln x}{\ln 2}} \right| = \frac{\ln 2}{\ln 10} \]

\item
	\[ \log_b \log_b x = \frac{\ln \log_b x}{\ln b} = \frac{\ln \frac{\ln x}{\ln b}}{\ln b} = \frac{\ln \ln x - \ln \ln b}{\ln b}\]

\item Da $b > 1$ ist das Minimum der Funktion bei $x = \frac{1}{e}$:
	\lstset{language=Mathematice}
	\begin{lstlisting}
		f[x_] = x Log[b,x];
		x/.Solve[f'[x] == 0, x]
		  {1/E}
		f''[1/E]
		  E/Log[b] (* > 0 für b > 1 *)
	\end{lstlisting}

\item Für $x > 1$ nimmt die Funktion ihr Minimum bei $b = e$ an:
	\begin{lstlisting}
		f[b_]= b Log[b,x];
		b/.Solve[f'[b] == 0, b]
		  {E}
		f''[E]
		  Log[x]/E (* > 0 für x > 1*)
	\end{lstlisting}

\item Das Minimum der Funktion wird bei $x = \log n$ erreicht:
	\begin{lstlisting}
		f[x_] = x n^{1/x};
		f'[x]//Simplify
		  {(n^(1/x) (x-Log[n]))/x}
		  (* Nullstelle bei x = Log[n] *)
		f''[Log[n]]
		  {E/Log[n]} (* > 0 *)
	\end{lstlisting}

\item Annahme $\log_{10} 2 = \frac{p}{q} \in \mathbb{Q}$ mit $p,q \in \mathbb{N}$ ($\log_{10} 2$ > 0, da
die Logarithmusfunktion streng monoton steigend ist und bei $1$ eine Nullstelle hat). Dann folgt:
\[ 10^{\frac{p}{q}} = 2 \Leftrightarrow 10^p = 5^p 2^p = 2^q \]
Die linke Seite hat Primteiler $5$, die rechte jedoch nicht; da die Primteilerzerlegung eindeutig ist
führt dies zu einem Widerspruch, also ist $\log_{10} 2 \not\in \mathbb{Q}$.
\end{enumerate}

