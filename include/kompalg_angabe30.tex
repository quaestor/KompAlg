Ein Permutation $\sigma=\sigma_1\,\sigma_2\, \sigma_3\,\ldots \,\sigma_{2n}$ der Zahlen
$\{1,2,\ldots,2n\}$ soll \emph{perfekt} heissen, wenn folgendes gilt:
\[
\sigma_1 < \sigma_3 < \sigma_5 < \cdots < \sigma_{2n-1}~~~\text{und}~~~
\sigma_1 < \sigma_2, \, \sigma_3 < \sigma_4 , \,   \ldots , \, \sigma_{2n-1} < \sigma_{2n}
\]
Anschaulich bedeutet das: ordnet man die $2n$ Zahlen $\sigma_1, \ldots,\sigma_{2n}$
folgenderma\ss en spaltenweise in einer Matrix an
\[
\begin{bmatrix}
\sigma_1 & \sigma_3 & \ldots & \sigma_{2n-1} \cr
\sigma_2 & \sigma_4 & \ldots & \sigma_{2n}
\end{bmatrix}
\] 
so sind alle Spalten und die die erste Zeile dieser Matrix monoton wachsende Folgen.
Die zweite Zeile muss nicht monoton wachsend sein.

Als Beispiel: für $n=2$ gibt es drei perfekte Permutationen:
\[
1234 \leftrightarrow \begin{bmatrix} 1 & 3 \cr 2 & 4 \end{bmatrix} ~~~
1324\leftrightarrow \begin{bmatrix} 1 & 2 \cr 3 & 4 \end{bmatrix} ~~~~
 1423\leftrightarrow\begin{bmatrix} 1 & 2 \cr 4 & 3 \end{bmatrix} 
\]

\begin{enumerate}
\item Stellen Sie durch Auflisten aller Möglichkeiten fest, dass es genau 15 perfekte Permutationen 
	von $\{1,2,3,4,5,6\}$ gibt.
\item Bezeichnet nun $p(n)$ die Anzahl der perfekten Permutationen von $\{1,2,\ldots,2n\}$,
	so zeigen Sie die Rekursion $p(n)= (2n-1)\cdot p(n-1)$ (mit dem Anfangswert $p(1)=1$).\\
	Hinweis: Betrachten sie die Matrixdarstellung, so muss die ''1''   in der Position $(1,1)$ stehen
	und in der Position $(2,1)$ kann eine der $2n-1$ Zahlen $2 \leq k \leq 2n$ stehen. Das sollte
	Sie auf ein Verfahren füren, wie man alle Möglichkeiten konstruiert.
\item Zeigen Sie
	\[
	p(n) = \frac{(2n)!}{2^n \cdot n!}~~~~~(n \geq 1).
	\]
\item Verwenden Sie die Formel von \textsc{Stirling}, um eine Aussage über das
	asymptotische Verhalten der Folge $(p(n))_{n  \geq 1}$ zu erhalten. Folgt daraus,
	dass diese Folge \emph{nicht} C-rekursiv ist?
\item Angenommen, Sie sollen ein Sortierverfahren für perfekte Listen entwerfen, also für 
	Listen, bei denen die Grössenverhältnisse der Listenelemente denen einer perfekten
	Permutation entsprechen. Geben Sie eine \emph{untere Schranke} für die
	Anzahl der Vergleichsoperationen an (worst-case und average-case), die \emph{jeder}
	hierfür geeignete Algorithmus benötigt.
\end{enumerate}

