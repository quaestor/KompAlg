%%%
% Angabe: Laufzeitenvergleich
%%%

\begin{flushenum}
\item
Ihnen liegen zur Lösung eines Problems drei
Algorithmen $\mathcal{A}, \mathcal{B}, \mathcal{C}$
vor, deren Laufzeit bei inputs der Grösse $n \in \mathbb{N}$
jeweils
\[
t_{\mathcal{A}}(n) = 4 \cdot n + 20~~~
t_{\mathcal{B}}(n) = n \cdot \log_2(n) + 2 \cdot n + 4~~~
t_{\mathcal{C}}(n) = n \cdot (n-1) + 4
\]
beträgt.
Welcher Algorithmus ist in Abhängigkeit von der
input-Grösse $n$ der Beste bzw. Schlechteste?

\item

Zur Berechung eines Problems stehen zwei Algorithmen $\mathcal{A}$ und
$\mathcal{B}$ zur Verfügung. Für deren jeweilige Komplexität (= Laufzeit auf
Instanzen der Grösse $n$) gelte
\[
t_\mathcal{A}(n) = \sqrt{n}~~~\text{bzw.}~~~t_\mathcal{B}(n) = 2^{\sqrt{\log_2 n}}
\]
\begin{flushalpha}
\item
Welcher der beiden Algorithmen ist asymptotisch besser (= schneller), d.h. gilt

$t_\mathcal{A}(n) \leq t_\mathcal{B}(n)$ oder 
$t_\mathcal{B}(n) \leq t_\mathcal{A}(n)$ für $n \rightarrow{\infty}$ ?
\item
Wo liegt der break-even-point, d.h. von welchem Wert der Instanzengrösse $n$ an ist der
asymptotisch bessere Algorithmus immer im Vorteil?
\item
Wie vergleichen sich die beiden Funktionen $t_\mathcal{A}(n)$ und $t_\mathcal{B}(n)$
in der Landau-Notation?
\end{flushalpha}

\end{flushenum}
 
