\section*{Laufzeitenvergleich}
\addcontentsline{toc}{subsection}{Laufzeitenvergleich}
\subsection*{Angabe}
%%%
% Angabe: Laufzeitenvergleich
%%%

\begin{flushenum}
\item
Ihnen liegen zur Lösung eines Problems drei
Algorithmen $\mathcal{A}, \mathcal{B}, \mathcal{C}$
vor, deren Laufzeit bei inputs der Grösse $n \in \mathbb{N}$
jeweils
\[
t_{\mathcal{A}}(n) = 4 \cdot n + 20~~~
t_{\mathcal{B}}(n) = n \cdot \log_2(n) + 2 \cdot n + 4~~~
t_{\mathcal{C}}(n) = n \cdot (n-1) + 4
\]
beträgt.
Welcher Algorithmus ist in Abhängigkeit von der
input-Grösse $n$ der Beste bzw. Schlechteste?

\item

Zur Berechung eines Problems stehen zwei Algorithmen $\mathcal{A}$ und
$\mathcal{B}$ zur Verfügung. Für deren jeweilige Komplexität (= Laufzeit auf
Instanzen der Grösse $n$) gelte
\[
t_\mathcal{A}(n) = \sqrt{n}~~~\text{bzw.}~~~t_\mathcal{B}(n) = 2^{\sqrt{\log_2 n}}
\]
\begin{flushalpha}
\item
Welcher der beiden Algorithmen ist asymptotisch besser (= schneller), d.h. gilt

$t_\mathcal{A}(n) \leq t_\mathcal{B}(n)$ oder 
$t_\mathcal{B}(n) \leq t_\mathcal{A}(n)$ für $n \rightarrow{\infty}$ ?
\item
Wo liegt der break-even-point, d.h. von welchem Wert der Instanzengrösse $n$ an ist der
asymptotisch bessere Algorithmus immer im Vorteil?
\item
Wie vergleichen sich die beiden Funktionen $t_\mathcal{A}(n)$ und $t_\mathcal{B}(n)$
in der Landau-Notation?
\end{flushalpha}

\end{flushenum}
 


%%%
% Lösung
%%%
\subsection*{Lösung}

\begin{flushenum}
\item
	Offensichtlich ist das asymptotische Verhalten der Algorithmen folgendes:
	\[ t_\mathcal{A} \in \mathcal{O}(n) \]
	\[ t_\mathcal{B} \in \mathcal{O}(n \log n) \]
	\[ t_\mathcal{C} \in \mathcal{O}(n^2) \]
	Und es gilt für große $n$, dass $t_\mathcal{A} < t_\mathcal{B} < t_\mathcal{C}$.
	Für kleine $n$ muss man dies jedoch genauer analysieren:
	\lstset{language=Mathematica}
	\begin{lstlisting}
		tA[n_] = 4 n + 20;
		tB[n_] = n Log[2,n] + 2n + 4;
		tC[n_] = n (n-1) + 4;
		n/.Solve[tA[n] == tB[n], n]//N
		n/.Solve[tA[n] == tC[n], n]//N
		n/.Solve[tB[n] == tC[n], n]//N
		  {10.9815}  (* A == B *)
		  {-2.21699,7.21699}  (* A == C *)
		  {0.1375,5.44491}  (* B == C *)
	\end{lstlisting}
	Damit erhält man folgende Abschätzungen:
	\begin{itemize}
		\item $n=0$: $t_\mathcal{B} = t_\mathcal{C} < t_\mathcal{A}$
		\item $1 \leq n \leq 5$: $t_\mathcal{C} < t_\mathcal{B} < t_\mathcal{A}$
		\item $6 \leq n \leq 7$: $t_\mathcal{B} < t_\mathcal{C} < t_\mathcal{A}$
		\item $8 \leq n \leq 10$: $t_\mathcal{B} < t_\mathcal{A} < t_\mathcal{C}$
		\item $n > 10$: $t_\mathcal{A} < t_\mathcal{B} < t_\mathcal{C}$
	\end{itemize}

\item 
	\begin{flushalpha}
	\item
		Bei dieser Aufgabe ist es sinnvoller von der Laufzeit der beiden Algorithmen den Logarithmus zu berechnen (solang $n \neq 0$,
		da die Ausdrücke so leichter zu vergleichen sind. Da der Logarithmus streng monoton ist,
		sind die gefundenen Nullstellen und der Größenvergleich identisch mit dem der ursprünglichen Laufzeiten.
		\[ \log_2 t_\mathcal{A}(n) = \frac{1}{2} \log_2 n \]
		\[ \log_2 t_\mathcal{B}(n) = \sqrt{\log_2 n} \]
		$\mathcal{B}$ ist damit der schnellere Algorithmus, da:
		\[ \lim_{n \rightarrow \infty} \frac{\log_2 t_\mathcal{B}}{\log_2 t_\mathcal{A}} = 
		   \lim_{n \rightarrow \infty} \frac{\sqrt{\log_2 n}}{\frac{1}{2} \log_2 n} =
		   \lim_{n \rightarrow \infty} \frac{2}{\sqrt{\log_2 n}} = 0 \]
	\item
		\[ t_\mathcal{A}(n) \lesseqgtr t_\mathcal{B}(n) \Leftrightarrow \sqrt{n} \lesseqgtr 2^{\log_2 n} 
		   \Leftrightarrow \frac{1}{2} \log_2 n \lesseqgtr \sqrt{\log_2 n} \Leftrightarrow \frac{1}{4} \log_2^2 n\lesseqgtr \log_2 n \]
		Hier ist zu sehen dass Gleichheit gilt wenn $n = 1$, ansonsten gilt weiter:
		\[ \Leftrightarrow \log_2 n \lesseqgtr 4 \Leftrightarrow n \lesseqgtr 16 \]
		Für $n=1$ und $n=16$ sind beide Algorithmen gleich schnell, für $1 < n < 16$ ist $\mathcal{A}$ schneller,
		für $n > 16$ ist $\mathcal{B}$ schneller.
	
	\item
		Mit Teilaufgabe 2.a) folgt $t_\mathcal{B} \in \hbox{o}(t_\mathcal{A})$.
	\end{flushalpha}
\end{flushenum}
