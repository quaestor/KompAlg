\section*{Eine zweite Folge zur \textsc{Fibonacci}-Rekursion}
\addcontentsline{toc}{subsection}{Eine zweite Folge zur \textsc{Fibonacci}-Rekursion}
\subsection*{Angabe}
%%%
% Angabe: Eine zweite Folge zur \textsc{Fibonacci}-Rekursion
%%%
Die \textsc{Fibonacci}-Folge $(F_n)_{n \geq 0} = (0,1,1,2,3,5,8,\ldots)$ (siehe Aufgabe: Rekursion und Induktion mit Fibonacci
und Abschnit 3.1.1 im Skriptum) genügt der linearen Rekursion
\begin{equation*}
x_{n} = x_{n-1} + x_{n-2}~~~(n \geq 2) \tag{$*$}
\end{equation*}
mit den Anfangswerten $F_0=0$ und $F_1=1$. Eine weitere bekannte Folge, die dieser Rekursion $(*)$
genügt, ist die \textsc{Lucas}-Folge\footnote{
Benannt nach den französischen Mathematiker Edouard \textsc{Lucas} (1842--1891), von dem
vemutlich auch die Bezeichnung der $F_n$ als \textsc{Fibonacci}-Zahlen stammt ---
fast 700 Jahre nach \textsc{Fibonacci}. 
\textsc{Lucas} hat sich viel mit Zahlentheorie befasst (Primzahltests!) und daneben mit seinen Büchern 
\emph{R\'ecr\'eations Math\'ematiques} (4 Bände) viel zur Popularisierung mathematischer
Denkweisen beigetragen. Von ihm (anonym veröffentlicht) stammen übrigens auch die 
bekannten \emph{Türme von Hanoi} und viele weitere populäre mathematische Probleme.}
\[
(L_n)_{n \geq 0} = (2,1,3,4,7,11,18,29, \ldots),
\]
die durch ihre Anfangswerte $L_0=2, L_1=1$ festgelegt ist.

\begin{flushenum}
\item Stellen Sie die Zahlen $L_n$ mit Hilfe von $\phi$ und $\widehat{\phi}$ dar (analog zur Aussage
	1. der Aufgabe: Rekursion und Induktion mit Fibonacci. Beachten Sie: die Folge $(L_n)_{n \geq 0}$ ist eine Linearkombination
	der beiden Folgen $(\phi^n)_{n \geq 0}$ und  $(\widehat{\phi}^n)_{n \geq 0}$.\\
	Notabene: $\widehat{\phi}^n$ ist als $\left( \widehat{\phi} \right)^n$ zu lesen.
\item Was ist $L_{n+1} L_{n-1} - L_n^2$? (analog  zur Aussage 2. der Aufgabe: Rekursion und Induktion mit Fibonacci
\item Was ist $\lim_{n \rightarrow \infty} \frac{L_{n+1}}{L_n}$ ~?
\item Wieviele Dezimalstellen hat die Zahl $L_n$ ~?
\item Die beiden Folgen $(F_n)_{n \geq 0}$ und $(L_n)_{n \geq 0}$ sind (offensichtlich!)
	linear-unabhängig im Vektorraum aller Folgen, die der Rekursion $(*)$ genügen. 
	Sie bilden also eine Basis dieses Raumes, ebenso, wie die beiden
	Folgen $(\phi^n)_{n \geq 0}$ und $(\widehat{\phi}^n)_{n \geq 0}$ eine Basis dieses
	Raumes bilden. Stellen Sie  $(\phi^n)_{n \geq 0}$ und $(\widehat{\phi}^n)_{n \geq 0}$
	als Linearkombinationen der beiden Folgen $(F_n)_{n \geq 0}$ und $(L_n)_{n \geq 0}$ dar.
\end{flushenum}




%%%
% Lösung
%%%
\subsection*{Lösung}
\begin{flushenum}
\item
	Da die Folge $(L_n)_n$ die gleiche Rekursion erfüllt wie die \textsc{Fibonacci}-Rekursion
	ist auch $L_n$ eine Linearkombination von $\phi^n$ und $\hat\phi^n$; mit den Startwerten
	können die Koeffizienten bestimmt werden:
	\[ L_0 = a + b = 2 \]
	\[ L_1 = a \cdot \phi + b\cdot \hat\phi = 1 \]
	Mit $\phi = \frac{1 + \sqrt{5}}{2}$ und $\widehat{\phi} = \frac{1 - \sqrt{5}}{2}$ ist das offensichtlich für $a = b = 1$ erfüllt.

\item 
	Da $L_n = L_{n-1} + L_{n-2}$ für alle $n \geq 2$ wie für die \textsc{Fibonacci}-Folge gilt, folgt nach Aufgabe 4. b):
	\[ L_{n+2}L_n  - L_{n+1}^2 = -1 \cdot (L_{n+1}L_{n-1} - L_n^2) \]
	Mittels Induktion folgt dann sofort, dass $L_{n+1}L_{n-1} - L_n^2 = (-1)^{n-1}5$:
	\begin{itemize}
		\item \textit{I.A.}: $n = 1$: $L_2L_0 - L_1^2 = 3 \cdot 2 - 1^2 = 5$
		\item \textit{I.V.}: Für ein $n \geq 1$ gilt:
			\[ L_{n+1}L_{n-1} - L_n^2 = (-1)^{n-1} 5 \]
		\item \textit{I.S.}: $n \rightarrow n+1$:
			\[ L_{n+1}L_n - L_{n+1}^2 = -1 \cdot (L_{n+1}L_{n-1} - L_n^2) \overset{I.V.}{=} (-1)^n 5 \]
	\end{itemize}

\item
	\[ \lim_{n \rightarrow \infty} \frac{L_{n+1}}{L_n} =
	   \lim_{n \rightarrow \infty} \frac{\phi^{n+1} + \hat\phi^{n+1}}{\phi^n + \phi^n} =
	   \lim_{n \rightarrow \infty} \frac{\phi^n \left(\phi + \hat\phi \left( \frac{\hat\phi}{\phi}\right)^n \right)}
	   				    {\phi^n \left( 1 + \left(\frac{\hat\phi}{\phi}\right)^n \right)} =
	   \frac{\phi + \hat\phi \cdot 0}{1 + 0} = \phi \]

\item Da $|\hat\phi^n| < 1$ kann man die Anzahl der Ziffern folgendermaßen abschätzen:
	\[ \log_ {10} L_n \approx \log_{10} \phi^n = n \log_{10} \phi \approx 0.21 \cdot n \]

\item
	\[ F_n = \frac{\phi^n - \hat\phi^n}{\sqrt{5}} \Leftrightarrow \sqrt{5} F_n = \phi^n - \hat\phi^n \]
	\[ L_n = \phi^n + \hat\phi^n \]
	Damit folgt unmittelbar:
	\[ \phi^n = \frac{L_n + \sqrt{5} F_n}{2} \]
	\[ \hat\phi^n = \frac{L_n - \sqrt{5} F_n}{2} \]
\end{flushenum}
