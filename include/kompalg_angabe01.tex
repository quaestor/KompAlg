%%%
% Angabe: Fingerübungen zum Rechnen mit Logarithmen
%%%

Bemerkung: Der sichere Umgang mit Logarithmen (und anderen elementaren
Funktionen der Mathematik, wie Polynome, Exponentialfunktion,
Winkelfunktionen) ist nicht Stoff der Veranstaltung,
sondern gehiört zum Handwerkszeug, dass man vorab beherrschen sollte.
Die nachfolgenden Aufgaben sind als Test zu verstehen: wer mit so etwas
ernsthaft Schwierigkeiten hat, hat etwas nachzuholen.
Diese Aufgaben werden nicht unbedingt alle in der Übungsgruppe behandelt.

Bezeichnung: für eine relle Zahl reelles $b>1$ und positives reelles $y$ 
bezeichnet $\log_b y$  den Logarithmus
von $y$ zur Basis $b$:
das ist die eindeutig bestimmte reelle Zahl $x$ mit $b^x = y$.
Logarithmen zur Basis $b=2$ werden oft einfach mit $\log y$ oder $\lg y$ bezeichnet, ``natürliche''
Logarithmen zur Basis $e$ mit $\ln y$.

\begin{flushenum}
\item
Ist $c$ eine weitere positive reelle Zahl, wie  drückt sich $\log_c y$
mittels Logarithmen zur Basis $b$ aus?
\item
In welchem Verhältnis stehen $\log_{10}2$ und $\log_2 10$ zueinander?
\item
Wieviele Binärstellen genügen in jedem Fall, 
um eine 14-stellige Dezimalzahl im Binärsystem darzustellen?
\item
Wieviele Dezimalstellen hat die Zahl\footnote{
N.B. Mit der Faktorisierung dieser Zahl in ihre zwei Primfaktoren
durch M. Morrison und J. Brillhard 
im Jahr 1970 begann die Neuzeit des Faktorisierens.} $2^{128}+1$ ?
\item
\"Uberzeugen Sie sich erst einmal durch Beispiele 
(mittels Taschenrechner, Maple, Logarithmentafel, \ldots) davon, 
dass für beliebiges $x >0$ die Zahlen
$\log_2 x$ und $\ln x + \log_{10} x$ verblüffend nahe beieinander liegen. 
Wie gross ist der relative Fehler?
\item
Wie drückt sich $\log_b \log_b x$ mittels natürlicher Logarithmen aus?
\item
An welcher Stelle nimmt die Funktion 
$x \mapsto x \cdot \log_b x$ für $x \in \mathbb{R}_+$
ihr Minimum an? 
\item
$x$ sei eine reelle Zahl $>1$. An welcher Stelle nimmt die Funktion 
$b \mapsto b \cdot \log_b x$ für $b \in \mathbb{R}_+$
ihr Minimum an? 
\item
Was ist das Minimum der Funktion 
$x \mapsto x \cdot \sqrt[x]{n}$ für $x \in \mathbb{R}_+, n \in \mathbb{N}$?
\item
Zeigen Sie, dass $\log_{10} 2$ keine rationale Zahl ist.

\end{flushenum}


