\section*{Aufgabe 13 - Eigenschaften der DFT}
\addcontentsline{toc}{subsection}{Aufgabe 13 - Eigenschaften der DFT}

\makeatletter
\def\Ddots{\mathinner{\mkern1mu\raise\p@\vbox{\kern7\p@\hbox{.}}\mkern2mu\raise4\p@\hbox{.}\mkern2mu\raise7\p@\hbox{.}\mkern1mu}}
\makeatother

\begin{enumerate}[1.]
	\item $DFT_N$ ist definiert als $DFT_N = [ \omega_N^{i \cdot j} ]_{i,j}$, wobei $\omega_N = e^{\imath \frac{2 \pi}{N}}$ 
	die erste $N$-te Einheitswurzel ist. Da $\omega_N^{i \cdot j} = \omega_N^{j \cdot i}$ ist $DFT_N = DFT_N^T$ eine symmetrische
	Matrix. Somit ist
	\begin{eqnarray*}
	DFT_N^2 &=& DFT_N \cdot DTF_N^T = DFT_N \cdot DFT_N = 
	\begin{bmatrix}
		\sum_{l = 0}^{N-1} \left(\omega_N^{i \cdot l} \cdot \omega_N^{l \cdot j}\right)
	\end{bmatrix}_{i, j} =
	\begin{bmatrix}
		\sum_{l=0}^{N-1} \omega_N^{l \cdot (i + j)}
	\end{bmatrix}_{i,j} = \\
	& = &
	\begin{bmatrix}
		N & \text{ für } i + j \equiv 0 \mod N \\
		0 & \text{ für } i + j \not\equiv 0 \mod N \\
	\end{bmatrix}_{i,l} = 
	[ N \cdot \chi_{i + j \equiv 0 \mod N} ]_{i,j}
	\end{eqnarray*}
	Die Matrix $DFT_N^2$ schaut damit folgendermaßen aus:
	\[ DFT_N^2 = 
	\begin{bmatrix}
		N      & 0      & \cdots & \cdots & \cdots & 0      \\
		0      &        &        &        & \Ddots & N      \\
		\vdots &        &        & \Ddots & \Ddots & 0      \\
		\vdots &        & \Ddots & \Ddots & \Ddots & \vdots \\
		0      & \Ddots & N      & \Ddots &        & \vdots \\
		0      & N      & 0      & \cdots & \cdots & 0      \\
	\end{bmatrix} \]
\end{enumerate}
