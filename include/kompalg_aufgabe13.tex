\section*{Eigenschaften der DFT}
\addcontentsline{toc}{subsection}{Eigenschaften der DFT}

\makeatletter
\def\Ddots{\mathinner{\mkern1mu\raise\p@\vbox{\kern7\p@\hbox{.}}\mkern2mu\raise4\p@\hbox{.}\mkern2mu\raise7\p@\hbox{.}\mkern1mu}}
\makeatother

%%%
% Angabe
%%%
\subsection*{Angabe}
\begin{flushenum}
\item Mit $\text{DFT}_N$  wird die Matrix der Diskreten Fouriertransformation
	der Ordnung $N$ bezeichnet.  Zeigen Sie, dass
	\[
	\text{DFT}_N^2 = \begin{bmatrix} N \cdot \chi_{i+j=0\, (N)} \end{bmatrix}_{0 \leq i,j < N}~~~
	\text{mit}~~
	\chi_{i+j=0\,(N)} =
	\begin{cases}
	1 &\text{falls}~i+j \equiv 0 \mod N \cr
	0 &\text{falls}~i+j \not\equiv 0 \mod N
	\end{cases}
	\]
	gilt und folgern Sie daraus
	\[
	\left| \, \det \text{DFT}_N \, \right|= N^{N/2}.
	\]

\item Für einen (Zeilen-)Vektor $\mathbf{a} = (a_0,a_1,a_2\ldots,a_{N-1}) \in
	\mathbb{C}^N$ sei $S(\mathbf{a}) = (s_0,s_1,s_2,\ldots,s_{N-1})=
	(a_0,a_{N-1}, \ldots,a_2,a_1)$ derjenige (Zeilen-)Vektor, der aus
	$\mathbf{a}$ durch Umkehrung und zyklischen Rechts-Shift  entsteht,
	d.h. es gilt $s_j = a_{N-j \mod N}$ für $0 \leq j < N$.  Für
	Spaltenvektoren gilt entsprechend $S(\mathbf{a}^t) = S(\mathbf{a})^t$.
	
	Zeigen Sie, dass die beiden Operationen $\mathbf{a} \mapsto
	\text{DFT}_N \cdot \mathbf{a}^t$ und $\mathbf{a} \mapsto S(\mathbf{a})$
	miteinander vertauschbar sind, d.h., es gilt
	\[
	\text{DFT}_N\cdot S(\mathbf{a}^t) = S(\text{DFT}_N \cdot \mathbf{a}^t)
	\]
	
	Tip: überlegen sie sich, welche Matrix die lineare Transformation
	$\mathbf{a} \mapsto \widetilde{\mathbf{a}}$ realisiert und verwenden
	Sie den vorigen Aufgabenteil.
 \item
Auf dem Vektorraum $\mathbb{C}^N$ definiert man das komplexe
\emph{Skalarprodukt}  für Vektoren ${\mathbf a} = (a_0 \ldots a_{N-1})$,
${\mathbf b} = (b_0 \ldots b_{N-1})$ durch
\[
\begin{array}{l} \langle {\mathbf a},{\bf b} \rangle := \sum\limits^{N-1}_{i=0}
	\overline{a}_i \cdot b_i  \end{array}
\]
Zeigen Sie:
\[
\langle {\bf a},{\bf b} \rangle = \frac{1}{N} \, \langle \text{DFT}_N \,{\bf a}^t,   \text{DFT}_N \,{\bf b}^t \rangle.
\]
Anmerkung: Dies ist das diskrete Analogon zu  der berühmten
\textsc{Parseval-Plancherel}-Gleichung im kontinuierlichen Fall der
Fouriertransformation.
\end{flushenum}

%%%
% Lösung
%%%
\subsection*{Lösung}
\begin{flushenum}
	\item Die Transformationsmatrix der diskreten Fouriertransformation der
		Länge $N$ ist definiert als $DFT_N = [ \omega_N^{i \cdot j}
		]_{i,j}$, wobei $\omega_N = e^{\imath \frac{2 \pi}{N}}$ die
		erste $N$-te Einheitswurzel ist. Da $\omega_N^{i \cdot j} =
		\omega_N^{j \cdot i}$ ist $DFT_N = DFT_N^T$ eine symmetrische
		Matrix. Somit ist
	\begin{equation*}
	\begin{split}
	DFT_N^2 & =  DFT_N \cdot DTF_N^T = DFT_N \cdot DFT_N \\
	& =
	\begin{bmatrix}
		\sum_{l = 0}^{N-1} \left(\omega_N^{i \cdot l} \cdot \omega_N^{l \cdot j}\right)
	\end{bmatrix}_{i, j} =
	\begin{bmatrix}
		\sum_{l=0}^{N-1} \omega_N^{l \cdot (i + j)}
	\end{bmatrix}_{i,j} \\
	& =
	\begin{bmatrix}
		N & \text{ für } i + j \equiv 0 \mod N \\
		0 & \text{ für } i + j \not\equiv 0 \mod N \\
	\end{bmatrix}_{i,l} = 
	[ N \cdot \chi_{i + j \equiv 0 \mod N} ]_{i,j}
	\end{split}
	\end{equation*}
	Die Matrix $DFT_N^2$ sieht damit folgendermaßen aus:
	\[ DFT_N^2 = 
	\begin{bmatrix}
		N      & 0      & \cdots & \cdots & \cdots & 0      \\
		0      &        &        &        & \Ddots & N      \\
		\vdots &        &        & \Ddots & \Ddots & 0      \\
		\vdots &        & \Ddots & \Ddots & \Ddots & \vdots \\
		0      & \Ddots & N      & \Ddots &        & \vdots \\
		0      & N      & 0      & \cdots & \cdots & 0      \\
	\end{bmatrix} \]
	Bei der Berechnung einer Determinante führt die Vertauschung zweier Zeilen
	(eine elementare Zeilenumformung) zu einem Wechsel des Vorzeichens und das
	Herausziehen eines Faktors $N$ aus {\bf einer} Zeile zu einem Faktor $N$ für die
	resultierende Determinate; damit erhält man für ein $n \in \mathbb{N}$:
	\begin{equation*}
	\begin{split}
	\det DFT_N^2 &= \begin{vmatrix}
		N      & 0      & \cdots & \cdots & \cdots & 0      \\
		0      &        &        &        & \Ddots & N      \\
		\vdots &        &        & \Ddots & \Ddots & 0      \\
		\vdots &        & \Ddots & \Ddots & \Ddots & \vdots \\
		0      & \Ddots & N      & \Ddots &        & \vdots \\
		0      & N      & 0      & \cdots & \cdots & 0      \\
	\end{vmatrix} = \\
	& =  (-1)^n \cdot \begin{vmatrix}
		N      & 0      & \cdots & \cdots & \cdots & 0      \\
		0      & N      &        &        &        & \vdots \\
		\vdots &       & \ddots &        &        & \vdots \\
		\vdots &        &        & \ddots &        & \vdots \\
		\vdots &        &        &        & \ddots & 0      \\
		0      & \cdots & \cdots & \cdots & 0      & N      \\
	\end{vmatrix} = \\
	&= (-1)^n N^N \det \mathds{1}_N = (-1)^n N^N \\
	\end{split}
	\end{equation*}
	Damit folgt wegen der Multiplikativität der Determinantenberechnung
	\[ | \det DFT_N^2 | = |\det DFT_N \cdot \det DFT_N| \]
	und somit:
	\[ | \det DFT_N | = \sqrt{|\det DFT_N^2|} = N^{N/2} \]

	\item Wie man leicht erkennen kann erfüllt die Matrix $DFT_N^2$ abgesehen von dem
	Faktor $N$ die lineare Transformation $S$, sodass
	\[ S(a^T) = \frac{1}{N} DFT_N^2 \cdot a^T \]
	gilt.
	Folglich ist $DFT_N$ und die lineare Transformation $S$ kommutativ:
	\begin{equation*}
	\begin{split}
	DFT_N \cdot S(a^T) &= DFT_N \cdot \frac{1}{N}DFT_N^2 \cdot a^T = DFT_N \cdot \frac{1}{N} DFT_N \cdot DFT_N \cdot a^T = \\
	&= \frac{1}{N}DFT_N^2 \cdot DFT_N \cdot a^T = S(DFT_N \cdot a^T)
	\end{split}
	\end{equation*}

	\item Für das komplexe Skalarprodukt $\langle a, b\rangle$ und eine
	Matrix $M$ gilt:
	\[ c = \langle a, M \cdot b\rangle = \langle M^\dagger \cdot a, b\rangle \]
	wobei $M^\dagger$ die adjungierte Matrix zu $M$ ist, also die Transponierte der komplex Konjugierten von $M$,
	i.e. $M = (m_{i,j})$ und $M^\dagger = (\overline{m}_{j,i})$.
	Dies kann man folgendermaßen durch Vertauschung der Summation zeigen:
	\begin{equation*}
	\begin{split}
	c &= \langle a, M \cdot b\rangle = \sum_{k=0}^{N-1} \overline{a}_k \cdot \left( M \cdot b \right)_k =
	\sum_{k=0}^{N-1} \overline{a}_k \cdot \sum_{i=0}^{N-1} M_{k,i} \cdot b_i =
	\sum_{k=0}^{N-1} \sum_{i=0}^{N-1} \overline{a}_k \cdot M_{k,i} \cdot b_i = \\
	&= \sum_{i=0}^{N-1} \left( \sum_{k=0}^{N-1} \left( \overline{a}_k \cdot M_{k,i}\right) \right) \cdot b_i =
	\sum_{i=0}^{N-1} \overline{\left( \overline{M}^T \cdot a \right)}_i
	\cdot b_i = \langle\overline{M}^T \cdot a, b\rangle = \langle M^\dagger
	\cdot a, b\rangle
	\end{split}
	\end{equation*}
	Aus der Vorlesung ist bekannt: 
	\[ DFT_N \cdot DFT_N^\dagger = DFT_N^\dagger \cdot DFT_N = N \cdot \mathds{1}_N \]
	Damit erhält man:
	\[ \frac{1}{N} \langle DFT_N \cdot a, DFT_N \cdot b\rangle =
	\frac{1}{N} \langle DFT_N^\dagger \cdot DFT_N \cdot a, b \rangle =
	\frac{1}{N} \langle N \cdot \mathds{1}\cdot a, b\rangle = \langle a, b\rangle \]
\end{flushenum}
