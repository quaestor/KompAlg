\section*{Aufgabe 4 - Fibonacci}
\addcontentsline{toc}{subsection}{Aufgabe 4 - Fibonacci}
\begin{paragraph}{a)}
  zu zeigen:
  \[ f_n = \frac{\Phi^n - \hat \Phi^n}{\sqrt{5}} \quad n \geq 0 \]
  \begin{itemize}
    \item Induktionsanfang:
    \begin{itemize}
      \item[$n = 0$]
        \[ f_0 = \frac{\Phi^0 - \hat \Phi^0}{\sqrt{5}} 
               = \frac{1 - 1}{\sqrt{5}} = 0 \]
      \item[$n = 1$]
        \[ f_1 = \frac{\Phi - \hat \Phi}{\sqrt{5}} 
               = \frac{1 + \sqrt{5} - 1 + \sqrt{5}}{2 \sqrt{5}} = 1 \]
    \end{itemize}
    \item Induktionsvoraussetzung: Die Behauptung gelte für $n$.
    \item Induktionsschritt: $n \rightarrow n + 1$
    
    Da $\Phi$ und $\hat \Phi$ die beiden Lösungen der quadratischen Gleichung
    $x^2 - x - 1 = 0$ sind, gilt offensichtlich $\Phi + 1 = \Phi^2$ sowie
    $\hat \Phi + 1 = \hat \Phi^2$.
%    \[ \Phi + 1 = \frac{3 + \sqrt{5}}{2} = \frac{6 + 2 \sqrt{5}}{4} 
%                = \frac{1 + s \sqrt{5} + 5}{4} = (\frac{1 + \sqrt{5}}{2})^2 
%                = \Phi^2 \]
%    analog gilt:
%    \[ \hat \Phi + 1 = \hat \Phi^2 \]
    
    Damit folgt:
    \[ f_{n+1} = f_n + f_{n-1} \overset{IV}{=} \frac{\Phi^n - \hat
       \Phi^n}{\sqrt{5}} + \frac{\Phi^{n-1} - \hat \Phi^{n-1}}{\sqrt{5}} = \]
    \[ = \frac{(\Phi^n + \Phi^{n-1}) - (\hat \Phi^n + \hat \Phi^{n-1})}{\sqrt{5}}
       = \frac{(\Phi^{n-1} \cdot \Phi + \Phi^{n-1}) - 
         (\hat \Phi^{n-1} \cdot \hat \Phi + \hat \Phi^{n-1})}{\sqrt{5}} = \]
    \[ = \frac{(\Phi^{n-1} (\Phi + 1)) - (\hat \Phi^{n-1} (\hat \Phi + 1))}{\sqrt{5}} 
       = \frac{\Phi^{n-1} \Phi^2 - \hat \Phi^{n-1} \hat \Phi^2}{\sqrt{5}} = \]
    \[ = \frac{\Phi^{n+1} - \hat \Phi^{n+1}}{\sqrt{5}} \]
    $\hfill \square$
  \end{itemize}
\end{paragraph}
\begin{paragraph}{b)}
  zu zeigen:
  \[ f_{n+1} f_{n-1} - f_n^2 = (-1)^n \quad n \geq 1 \]
  \begin{itemize}
    \item Induktionsanfang:
    \begin{itemize}
      \item[$n = 1$] \[ f_2f_0 - f_1^2 = 1 \cdot 0 - 1^2 = (-1)^1 \]
    \end{itemize}
    \item Induktionsvoraussetzung: Die Annahme gelte für $n$.
    \item Induktionsschritt: $n \rightarrow n + 1$

    Mit dem Zusammenhang $f_{n} = f_{n-1} + f_{n-2}$ für alle $n \geq 2$ folgt:
    \[ f_{n+2}f_n - f_{n+1}^2 = (f_{n+1} + f_n)(f_{n+1} - f_{n-1}) - f_{n+1}^2
                              = -f_{n+1}f_{n-1} + f_{n+1}f_n - f_nf_{n-1} = \]
    \[ = -1(f_{n+1}f_{n-1} - (f_{n+1}f_n - f_nf_{n-1})) 
       = -1(f_{n+1}f_{n-1} - (f_n(f_{n+1} - f_{n-1}))) = \]
    \[ = -1(f_{n+1}f_{n-1} - f_n^2) \overset{IV}{=} -1(-1)^n = (-1)^{n+1} \]
    $\hfill \square$
  \end{itemize}
\end{paragraph}
\begin{paragraph}{c)}
  zu zeigen:
  \[ \lim_{n \rightarrow \infty} \frac{f_{n+1}}{f_n} = \Phi \]
  Vorbemerkung:
  \[ \lim_{n \rightarrow \infty} (\overbrace{\frac{\hat \Phi}{\Phi}}^{< 1})^n = 0 \]
  Beweis:
  \[ \lim_{n \rightarrow \infty} \frac{f_{n+1}}{f_n} 
     = \lim_{n \rightarrow \infty} \frac{\Phi^{n+1} - 
       \hat \Phi^{n+1}}{\Phi^n - \hat \Phi^n} = \]
  \[ = \lim_{n \rightarrow \infty} \frac{\Phi^{n+1}}{\Phi^n - \hat \Phi^n} - 
       \lim_{n \rightarrow \infty} \frac{\hat \Phi^{n+1}}{\Phi^n - \hat \Phi^n} = \]
  \[ = \lim_{n \rightarrow \infty} \frac{\Phi^n \cdot \Phi}{\Phi^n (1 - \frac{\hat \Phi^n}{\Phi^n})} - 
       \lim_{n \rightarrow \infty} \frac{\frac{\hat \Phi^{n+1}}{\Phi^{n+1}}}
                                        {\Phi^{n+1}(\Phi^{-1} - \frac{\hat \Phi^n}{\Phi^{n+1}})} = \]
  \[ = \lim_{n \rightarrow \infty} \frac{\Phi}{1 - (\frac{\hat \Phi}{\Phi})^n} - 
       \lim_{n \rightarrow \infty} \frac{(\frac{\hat \Phi}{\Phi})^{n+1}}
                                        {\Phi^{-1} (1 - (\frac{\hat \Phi}{\Phi})^n)} 
     = \Phi - 0 = \Phi \] $\hfill \square$
  \begin{table}[h!b!p!]
  \caption{Wertetabelle:}
  \begin{center}
  \begin{tabular}{r|cccccccccc}
  $n$                   & 1  &  2  &  3  &  4  &  5  &  6  &  7  &  8  &  9  & 10 \\
  \hline
  $\frac{f_{n+1}}{f_n}$ & 1  &  2  & $\frac{3}{2}$ & $\frac{5}{3}$ & $\frac{8}{5}$ & $\frac{13}{8}$ & $\frac{21}{13}$ & $\frac{34}{21}$ & $\frac{55}{34}$ & $\frac{89}{55}$
  \end{tabular}
  \end{center}
  \end{table}

  \parindent 0pt
  Zahlengerade: \\[0.5em]
  \begin{tikzpicture}[scale=1,cap=round]
  \tikzstyle{axes}=[]
  \begin{scope}[style=axes]
    \draw[loosely dotted] (-2,0) -- (-1,0);
    \draw (-1,0) -- (7,0);
    \draw[loosely dotted] (7,0) -- (9,0);
    \draw[->] (9,0) -- (12,0);
    \foreach \x/\xtext/\xname in {0.0/1.5/A, 6.4/1.66/C, 4.76/1.615/E}
      \draw[xshift=\x cm] (0pt,2pt) -- (0pt,-2pt) node[above=3pt,fill=white] (\xname) {$\xtext$};
    \foreach \x/\xtext/\xname in {4.0/1.6/B, 5.00/1.625/D, 10.0/2.0/F}
      \draw[xshift=\x cm] (0pt,2pt) -- (0pt,-2pt) node[below] (\xname) {$\xtext$};
    \draw[->] (F.south) to [out=270,in=270] ($(A.south) +(0.0,-5pt)$);
    \draw[->] (A.north) to [out=90,in=90] (C.north);
    \draw[->] ($(C.south) +(0.0,-5pt)$) to [out=270,in=270] (B.south);
    \draw[->] ($(B.north) +(0.0,5pt)$) to [out=90,in=90] ($(D.north) +(0.0,15pt)$);
    \draw[->] (D.south) to [out=270,in=300] ($(E.south) +(0.0,-15pt)$);
  \end{scope}
  \end{tikzpicture}
\end{paragraph}
\begin{paragraph}{d)}
$f_n$ wächst wie $\frac{\Phi^n}{\sqrt{5}}$ (siehe Vorlesung)
\[ \Rightarrow \frac{\Phi^n}{\sqrt{5}} = b^x \] wobei $b$ die Basis und $x$ die Anzahl der Stellen ist.
\[ x = \log_b \frac{\Phi^n}{\sqrt{5}} = n \log_b \Phi - \log_b \sqrt{5} \]
\begin{itemize}
  \item $\log_{10} \Phi \approx 0.20898,\quad \log_{10} \sqrt{5} \approx 0.34949\quad \Rightarrow 0.209\cdot n - 0.35 \textnormal{ Dezimalziffern}$
  \item $\log_{2}  \Phi \approx 0.69424,\quad \log_{2} \sqrt{5} \approx 1.16096\quad \Rightarrow 0.694\cdot n - 1.161 \textnormal{ Binärziffern}$
\end{itemize}
\begin{table}[h!b!p!]
\begin{center}
\begin{tabular}{r|cccc}
& $f_{100}$ & $f_{200}$ & $f_{300}$ & $f_{400}$ \\
\hline
Dezimalziffern &  21 & 42 & 63 & 84 \\
Binärziffern   &  69 & 138& 208& 277
\end{tabular}
\end{center}
\end{table}
\end{paragraph}
